%!TEX root = ../thesis.tex

% 12 june draft complete

The ethical parameters of the investigation were informed by consideration of:

\begin{enumerate}
    \item International research ethics literature that deals with \gls{CMC} and \glspl{OSG}
    \item The medium and situation factors unique to the \glslink{Forum}{Bipolar Forum}
    \item The \emph{Australian National Statement on Ethical Conduct in Human Research 2007 (updated May 2015)} \nocite{national_health_and_medical_research_council_the_australian_research_council_and_the_australian_vice-chancellors_committee_national_2015}
\end{enumerate}
%
The guidelines in the National Statement take precedence over the broader international literature. That said, the National Statement provides little guidance on \gls{CMC}, which often challenges existing notions of privacy, anonymity and informed consent. As such, consideration of ethical concerns raised in the relevant literature is prudent. Strategies employed to minimise risk to participants are also described.

\subsection{Considering \glsfmtshort{CMC}\slash \glsfmtshort{OSG} data}

Much has been written about the ethics of researching \gls{CMC} \cite{boyd_social_2007-1,ess_internet_2007,eysenbach_towards_2000,hewson_ethics_2015,landert_private_2011,stevens_public_2015,walther_research_2002}. Common is a recognition that different \glspl{mode} require different ethical considerations: forms of \gls{CMC} that can be easily connected to real\hyp{}world identities (due to the use of real names, profile photos, etc.) necessitate different protocols from forms that are highly anonymised. Similarly, the level of privacy and anonymity is not necessarily the same for all \glspl{mode}: on YouTube, video bloggers show their faces and often give their real names, while commenters on the clips are generally highly anonymised.

Another factor influencing ethical decisions is the size of the dataset. In the case of very large quantities of user\hyp{}generated natural language, it may become impossible for researchers to check that data has been fully anonymised. For these reasons, literature concerned with the ethics of using Facebook\slash chat data \cite[e.g.][]{hudson_``go_2004,zimmer_but_2010} or small\hyp{}scale, qualitative datasets \cite[e.g.][]{eysenbach_ethical_2001,roberts_ethical_2015} are not particularly relevant here.

The use of online \glspl{forum} as data sources has been a contentious issue. Though most acknowledge that \glspl{forum} are `public', the fact that \glslink{post}{contributions} are often authored in private spaces such as bedrooms may influence the candidness of texts \cite{hewson_ethics_2015}. Others have questioned the usefulness of the `public\slash private' binary in online contexts more generally \cite{lange_publicly_2007}. When compared to face\hyp{}to\hyp{}face data, researchers also know less in general about the participants, making it difficult to assess potential harm on an individual basis. In quantitative studies of thousands of users, spanning over a decade of \glslink{post}{contributions}, this becomes an impossibility.

Another issue is the potential for quotations, even when stripped of contextual information, to be linked back to their source: since texts in online \glspl{forum} are indexed in search engines, simply entering quotes into a search engine will often quickly uncover the original \gls{thread}. There, one can access the user's profile, and in some cases even send the user a private message. One strategy for avoiding this issue has been to paraphrase or translate quotes \cite{stommel_use_2011,vayreda_social_2009}. In an investigation of subtle changes in linguistic choices, however, this sacrifices the authenticity of the data.

%Related is the issue of what constitutes anonymity for online forum users. At what point do we say that a participant has been de-anonymised? The public profile of the user? The user's email address? The user's username? As with the issue of public vs. private information, contemporary debate frames anonymity as a continuum \cite{nagel_anonymity_2015}, rather than a binary.

%The level of privacy and anonymity is not necessarily the same for all interactants: on YouTube, video bloggers show their faces and often give their real names, while commenters on the clips are generally highly anonymised.

\textcite{hewson_ethics_2015} and \textcite{markham_ethical_2012} argue that flexible, bottom\hyp{}up, contextually sensitive parameters should be created for any study of \gls{CMC}. Accordingly, below, I summarise issues of privacy, anonymity and informed consent with respect to the specific medium and situation factors of the forum.

\subsubsection{Contact with participants} 

Participants were not contacted at any stage before, during or after the study. Therefore, none of the data was researcher\hyp{}elicited, and no non\hyp{}publicly available data were generated for the project. The National Statement includes `damage to social networks' (p.~13) as a kind of harm that research can do to participants. Not contacting \gls{Forum} \glslink{member}{users} therefore minimises harm by preserving an existing social structure as\hyp{}is. At the same time, the decision not to contact participants made it impossible to obtain informed consent. The highly anonymised nature of the board, as well as the longitudinal focus of the case study, however makes obtaining such consent from all \glslink{member}{contributors}, or even a plurality thereof, impossible \cite{kaufman2016producing,stommel_use_2011}. %In the case of this study, however, the issue is moot: as explained below, the National Statement exempts the study from the process of obtaining participants' consent.

\subsubsection{Anonymity and privacy}

Online spaces vary in the extent to which users' contributions are publicly accessible, and in how much users can reveal about their identity. In the \glslink{Forum}{Bipolar Forum}, \glslink{member}{users} protect their anonymity: they use nicknames, and do not contribute information that could identify them offline, such as real names, addresses, social security numbers, or photographs. In both the account creation and posting guidelines, \glslink{member}{users} are reminded to keep \gls{Forum} contents anonymous, and to assist in the anonymisation of others:

\begin{quote}
\small \singlespacing 
Do not register your surname or put identifying info in your profile or signature: Your username, profile, signature and messages must not identify you to readers or contain any part of your email address, blog or website. % http://www.healthboards.com/boards/faq.php?faq=faq_hb#faq_new_faq_item

Please only use first names of members. To protect anonymous use of the site do not include surnames. Use the listed first name or the username. % http://www.healthboards.com/boards/faq.php?faq=faq_hb#faq_poliguid

\end{quote}
%
Moderators also have the power to censor content that may de\hyp{}anonymise the user. Ultimately, \glslink{member}{users} appear to adhere to these rules: searching the \gls{corpus} for addresses, email addresses and phone numbers did not turn up a single instance that needed anonymisation, nor did any such information emerge throughout the course of the investigation.

%Again, the National Satement argues that exposing Participants' crimes can constitute harm `including discovery and prosecution of criminal conduct.' (p.~13)

Finally, all analysed data is publicly available. Though some researchers have problematised academic use of user\hyp{}generated \gls{CMC} \cite[e.g.][]{eysenbach_towards_2000,zimmer_but_2010}, such critiques centre on the notion that participants are unaware of the publicly available nature of their contributions. This is not a reasonable argument in the case of the \glslink{Forum}{Bipolar Forum},\endnote{Users' expectation of privacy is more likely to be an ethical issue in modes such as online chat, where chat messages appear to vanish after new messages arrive, and where chat transcripts are not searchable online, or connected through hyperlinks to a user's profile.} for a number of reasons:

\begin{enumerate}
    \item Users can see how many others are currently online, and how many people have viewed each thread.
    \item The main page invites users to \texttt{Subscribe} for an automatic bulletin of popular posts.
    \item Every page contains a \texttt{Search} button, showing that all posts are archived and indexed.
    \item Users' profiles are explicitly called \texttt{Public Profile Pages}. Help pages for Public Profile creation remind users that what they add is `publicly available'. %http://www.healthboards.com/boards/faq.php?faq=vb3_user_profile#faq_vb3_public_profile
\end{enumerate}
%
To argue that \gls{Forum} \glslink{member}{users} are not aware of the public nature of their interactions is to argue that \glslink{member}{users} fundamentally misunderstand the design and function of the community. No evidence supporting the idea that \glslink{member}{users} have such a misunderstanding was found over the course of the investigation.

% The linguistic content of their contributions consistently demonstrates that users are aware that their posts are read by strangers

\subsection{Interpreting the National Statement}

Under the National Statement, application for review and clearance from a Human Research Ethics Committee is required when the research carries more than a low risk of harm, distress or, at minimum, discomfort, to participants.\endnote{Under the National Statement, forum users qualify as participants, despite their not being contacted, or even being aware of the fact that their data is being analysed.} Exempted from the review process, however, is research that:

\begin{quote}
\small \singlespacing
\begin{enumerate}
    \item is negligible risk research; and
    \item involves the use of existing collections of data or records that contain only non\hyp{}identifiable data about human beings (p.~70).
\end{enumerate}
\end{quote}
%
The case study qualifies as negligible risk, as any potential risks for \gls{Forum} \glslink{member}{contributors} do not rise to the level of discomfort. The data qualifies as non\hyp{}identifiable, as they `have never been labelled with individual identifiers' (p.~27). More specifically, the \gls{Forum} data constitutes a subset of the non\hyp{}identifiable data class, which 

\begin{quote}
\small \singlespacing
are those that can be linked with other data so it can be known that they are about the same data subject, although the person's identity remains unknown (p.~27).
\end{quote}
%
It is indeed possible to connect different contributions to a single author due to the username (and potentially, the linguistic content of the contribution). This, however, cannot be connected to individual identifiers.

Because the case study is exempted from the process of ethics review, and due to the consideration of broader literature and site\hyp{}specific factors, an application for review was not made.

\subsubsection{Participants with mental health issues}

The majority of users of the \gls{Forum} may be classified as having a mental illness. The National Statement mandates that the `distinctive vulnerabilities' of such people must be taken into account, while also protecting the entitlement of such people to participate in research. Researching those with mental health issues may also involve differing guidelines for obtaining consent. According to the Statement, however, in cases where `research uses collections of non\hyp{}identifiable data and involves negligible risk', the need for informed consent is waived.

%It is also notable that the overall topic of the forum is less sensitive than topics that deviate from mainstream medical advice (such as pro-anorexia forums) and\slash or are potentially illegal, such as drug use or pedophilia forums, both of which have been studied using similar methods.

\subsection{Minimising risk}

Under the National Statement, `researchers have an obligation to minimise the risks to participants' (p.~14). To minimise any potential risk of privacy invasion, \glslink{member}{users'} public profiles were excluded from data collection, and usernames have been paraphrased throughout the thesis. Any part of the \gls{Forum} restricted to registered \glspl{member} was not accessed or considered in the analysis. The \gls{corpus} therefore includes only text that is freely accessible by navigating the \glslink{Forum}{Bipolar Forum}.

To ensure that no social structure is disturbed, and to ensure \glslink{member}{contributors'} privacy, all texts chosen for sustained, qualitative analysis are authored by \glslink{member}{users} who are now inactive within the \glslink{Forum}{community}, having not \glslink{post}{posted} in the past year.