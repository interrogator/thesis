%!TEX root = ../thesis.tex

\chapter{Future directions} \label{chap:futuredirections}

The previous chapters of the thesis have demonstrated the utility of \gls{CL} approaches to discourse analysis as a means of understanding consumer healthcare talk. The \gls{corpus} analysis uncovered subtle changes in linguistic choices over the course of membership in the community, and over the history of the \gls{Forum} itself. New and veteran \glslink{member}{users}, while contributing to a shared Field of Discourse, construe participants and processes within the Field differently. Interpersonal dimensions of language use are also subject to change, with shifts from demanding to providing information, and from health information to social support as the central semiotic commodity being most commonly exchanged.

More generally, the case study demonstrated new methods with which delicate features of the lexicogrammar in corpus data can be extracted and quantified. Using the proposed approach, workflows can be highly automated and recursive, producing usefully reduced summaries of linguistic behaviour in corpora and subcorpora. %that linguists can readily abstract to form an understanding of discourse and semantics in the dataset.
At the same time, however, the investigation highlighted a number of limitations inherent to the methodology. Most centrally, the lack of dedicated parsing at the stratum of discourse-semantics means that discursive or semantic research questions can only be approached via lexicogrammar, relying on theory and human judgement to form a coherent account of the entire content stratum of language. A second key limitation is the present lack of understanding of how we can best connect language features of computer\hyp{}mediated, intra--consumer health communication to real\hyp{}world health outcomes. For this reason, it is difficult to translate case study findings into actionable suggestions for clinical policy.

In this chapter, I focus on future research that can draw upon the methods and findings of the case study. The chapter is divided into two parts. First, I discuss the contributions of the case study to the three research areas of \gls{CL}, \gls{SFL} and healthcare communication. Second, I discuss the potential for discursively oriented consumer--consumer healthcare communication to be used within medical computational linguistic workflows in order to improve treatment outcomes. This discussion culminates in a research agenda oriented toward \textbf{(a)} better understanding of the usefulness of intra-consumer health talk for improving treatment outcomes; and \textbf{(b)} working toward extraction of useful discourse-semantic information from healthcare-institutional texts.

\section{New tools and methods for corpus\slash discourse research}

%\paragraph{Situating the Bipolar Forum register within the healthcare institution}

%With the register of the Bipolar Forum sketched, it can be mapped against contributions made by other scholars. 

%We can add a variable to the registerial map that captures the absent participants who may be more freely discussed.

%\textbf{Figure here with dimension of absent participants}
% adding absent participants

\begin{comment}

\section{A research agenda for computational healthcare discourse research}
\label{sec:research_agenda}

The second major purpose of this chapter is to highlight the potential usefulness of lay consumer health discourse, and its computational analysis, in the emerging field of medical computational linguistics. I argue that the kinds of things that can be learned through automated analysis of intra-consumer health discourse may complement cutting-edge methods for extracting semantic information from texts in the medical domain, and in linking this information to treatment outcomes.













\subsection{Rationale}

The ever-increasing amount of digital text in hospitals and clinics has opened up the potential for \gls{NLP} to enter healthcare institutions and healthcare research. Because of the high-stakes of many potential applications, such as automated diagnosis from radiologists' reports, or prediction of compliance with medication regimens, and because the use of \gls{NLP} in healthcare contexts is nascent, current applications of NLP generally only supplement existing decision-making practices \cite{maddox_natural_2015}. In research contexts, however, \gls{NLP} has shown a great deal of promise. 





Current applications of \gls{NLP} are typically low-stakes, and often supplementary to 




%Online data has applications in its own right,

%but also, the kinds of methods developed on online text can eventually be applied to medical free text, such as clinicians' notes, consumers' elicited narratives\slash interactions, or reports from specialists.




%It also cannot be ignored that users of dedicated offline and online support communities report substantial and often tangible benefits \cite{park_automatically_2015}.

%Qualitative healthcare communication research often proposes guidelines for clinical practice that reflect the findings of investigations. An acknowledged difficulty is that the kinds of evidence required for policy change is often at odds with what qualitative research returns.

%The replicability, scale and accuracy of automated analysis of language in healthcare communication seen in the case study can do much to ameliorate this shortcoming,

\subsubsection{Patient discourse and health outcomes}

\subsubsection{Underexploited patient-centred data}

\subsubsection{Linking experiences across domains}

\subsection{Toward needed tools}
%
%Most critical tasks in NLP have improved dramatically since the beginning of the 21st century, with speed and accuracy expected to improve 

%Renewed interest in neural networks and machine learning are seen as playing a large role in future advances.

\subsubsection{Modelling register computationally}

%A computational model of healthcare registers permits register-specific parsing.

%Attempts to computationally model register typically split along metafunctional lines proposed in \gls{SFL} (Teich).

%This makes a particular recursive approach possible: texts can be annotated with surface-level features, which can inform a classification of the text within the pre-existing computational model of registers. From here

%The final output of the parsing algorithm can feed back into the register model for future classificatory work.

\subsubsection{Annotation of discourse-semantics}

%The key limitation of the case study is that insights into discourse and semantics are gained only via lexicogrammar. Moreover, the investigation centres on congruent realisations, and on realisations made between the group and clause ranks.

%I propose that discourse-semantic features of texts can be accurately annotated onto texts, so long as the documents can be fit within an accurate computational model of register that facilitates domain-specific parsing.

\subsection{Applications}

%In this section, I outline three possible use-cases for the proposed methodology.

\subsubsection{Measuring patient centredness}

%Patient-centredness is an important construct in contemporary medicine, where professionals are mindful of patients and their journeys through the healthcare institution. This addresses a serious health risk within the previous paradigm, where little effort is made to link consumers' encounters, and to ensure that consumers' concerns are transmitted to those who can ameliorate potential risk points. This has been developed further within an SF terminological framework as \emph{relationship-centred care}, in which increased attention is paid during health encounters to constructive and collaboratory interpersonal meaning-making between healthcare professional and consumer.

%A possible use-case of this kind of workflow would be to measure levels of patient-centredness, and to map these levels to health outcomes. This could be especially useful in longitudinal contexts.

\subsubsection{Profiling consumers}

%We could also use annotated data, SF theory and machine learning to profile site users, in order to map personality, socio-economic distinctions and the like to health outcomes observed in formal healthcare institutions \cite{argamon_automatically_2009}.

\subsubsection{Staging interventions}

%The conceptualisation of linguistic corpora as static bodies of text is giving way to an understanding that corpora can be temporary, self-updating, web-based.

%It is possible to mine contributions as they stream in

%In this way, users' language use can be monitored in something near real-time.

%Rapid change in language use, or the use of language containing potentially harmful meanings (suicidal ideation, for example) could be detected, and a healthcare professional mobilised to begin a conversation with the user.

\subsubsection{Summary of applications}

%In each case, tools that can interface with high-level frameworks for annotating text, extracting relevant features, and performing statistical analysis are invaluable.


\section{Summary}
% \bibliography{references2}

\end{comment}