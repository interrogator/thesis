%!TEX root = ../thesis.tex

% revised june 11, to leave alone for now

\chapter{Introduction} \label{chap:intro}

In this chapter, I provide an outline of the investigation and contemporary literature from relevant research areas. This is followed by a statement of the problem and an explanation of the approach taken to address this problem. The main contributions of the thesis are summarised, and the structure of the thesis outlined.

\section{Context of the thesis}

It is well\hyp{}established that the Internet is a popular source and vast repository of health information, the provision of which may take place via a diverse set of \glspl{mode}. These \glspl{mode} range from dedicated websites for particular health conditions or health organisations, in which healthcare professionals author content targeting healthcare \glspl{consumer}, to \glspl{mode} oriented toward intra\hyp{}consumer interaction, such as social networking, web \glspl{forum} or wikis \cite{sillence_why_2013}. Of particular interest to linguistic research in the past two decades has been interactional sites of health information exchange such as online support groups (\glspl{OSG}): websites (or parts thereof) where \glslink{member}{users} can \gls{post} and reply to \glspl{thread}. \glspl{OSG}, and online \glspl{forum} in general, are a well\hyp{}known \gls{mode} of \gls{CMC}. In many cases, \glsxtrlongpl{OSG} can be viewed without logging in, and their \glspl{thread} are returned in search engine queries. \glslink{post}{Posting} and replying to \glspl{thread}, however, usually involves creating an account. \Glspl{forum} vary widely in terms of their medium (i.e. technological) and situation (i.e. socially prescribed) affordances \cite{herring_faceted_2007}: some have strict moderation and can enforce bans; some allow \glslink{member}{users} to embed images and videos into \glspl{post}; some allow \glslink{member}{users} to create profiles and send private messages \cite{morzy_analysis_2012}. Having remained in use since their beginnings in the 1990s, these \glspl{forum} have a long research history \cite[e.g.][]{sharf_communicating_1997}. That said, there is growing evidence that the popularity of online \glspl{forum} is in decline, with consumers instead receiving information and support from social networking sites (e.g. \emph{Facebook}), link\slash content aggregation platforms (e.g. \emph{Reddit}) or any of countless dedicated mobile apps (for diet and weight management, exercise, smoking cessation, and so on).

% conflict here: gruba is ok with using an image, robyn is not
%\begin{figure}[htb]
%\centering
%\addvbuffer[12pt 8pt]{\includegraphics[width=0.65\textwidth]{../images/forum.png}}
%\caption{The main page of the \emph{Bipolar Forum}}
%\label{fig:forum}
%\end{figure}


% what about health info benefits

%\cite{yan2015good} cites a number of benefits of social support for health

When compared with face\hyp{}to\hyp{}face support groups, \glspl{OSG} may have unique benefits and consequences for \glslink{member}{users'} health. In terms of benefits, \gls{forum} \glslink{member}{users} generally have round\hyp{}the\hyp{}clock access to a global community, facilitating larger groups and constant support \cite{stommel_online_2010,stommel_use_2011}. This support has been linked to improved understanding of illness, the development of coping strategies, reduced anxiety and depression, and an increasing confidence in health professionals \cite{mulveen_interpretative_2006,swan_sharing_2010,manchaiah_use_2013,yao_impact_2015}. Furthermore, the degree of anonymity in such environments has sometimes been found to lead to a disinhibiting effect, encouraging honest discussion \cite{mo_are_2013} with qualitative differences from professional--consumer and\slash or face\hyp{}to\hyp{}face interactions \cite{maclean_forum77:_2015}. Of concern to some researchers, however, has been the potential spread of misinformation due to non\hyp{}expert advice \cite{ziebland_how_2004}. Regular \gls{forum} \glspl{member} may gain expert status within \glspl{OSG} despite a lack of formal medical credentials \cite{hardey_doctor_1999,thompson_credibility_2012}. Such concerns are compounded in situations involving vulnerable participants who may not have the ability to make sound decisions regarding their own health. Finally, \gls{forum} cultures may have the effect of normalising mental health issues such as suicidal ideation and eating disorders, and may encourage \glslink{member}{users} to exhibit symptoms or obtain diagnoses for the purposes of legitimating themselves socially within the community \cite{horne_doing_2009,vayreda_social_2009}. These issues have been hypothesised to be the result of the dual function of \glspl{OSG} as sites for both health information and social support exchange \cite{nambisan_information_2011,attard_thematic_2012}.

\subsection{Language use in online support groups} \label{sect:intro-lang-in-osg}

In \glspl{OSG}, language is the dominant resource through which meanings are made. Accordingly, \gls{OSG} research almost always involves some level of analysis of the language use of \gls{forum} \glspl{member}, with or without explicit reference to linguistic theory. Within discourse\hyp{}analytic \gls{OSG} research, key areas of interest include (\emph{i}) member roles, (\emph{ii}) advice provision, (\emph{iii}) legitimation and (\emph{iv}) socialisation. In terms of \gls{member} roles and advice, though some early research in this area has been motivated by a concern that non\hyp{}professional `experts' may provide incorrect or harmful information, more contemporary findings generally suggest that the advice provided by veteran members to newcomers is often in line with mainstream biomedical norms \cite{vayreda_social_2009}, and commonly mundane (i.e. \emph{Go and consult with your doctor}) in nature \cite{smithson_problem_2011}. Legitimation research has for the most part focussed on the ways in which newcomers construct an identity and message that warrants useful responses from others \cite{galegher_legitimacy_1998,west_facework_2010}. Depending on the community, legitimation strategies have been found to vary widely: newcomers may emphasise the severity and uniqueness of their case \cite{varga2014grieving} or stress their inexperience and need for guidance. The third main focus within \gls{OSG} research is socialisation---that is, how \glslink{member}{users} learn through participation in meaningful social interaction with more experienced members of groups \cite{ochs_socialization_1991}. \textcite{lee_new_2014}, for example, approach socialisation from a member\hyp{}life\hyp{}cycle perspective, arguing that \glspl{member} transition through a number of roles during their time within the online community, and that each role has accompanying needs and responsibilities. Newcomers have strong needs for both information and social support, but may not contribute due to the potential for loss\hyp{}of\hyp{}face if information they provide is judged by experts to be incorrect \cite{fuller_innovation_2007} or at odds with community\hyp{}specific values \cite{weber_missed_2011}. At later stages in the \gls{member} life\hyp{}cycle, \glslink{member}{users} become less anxious about producing content, but lose the motivation to seek out information or support \cite{lee_new_2014}. Within this literature, lacking so far have been accounts of the longitudinal evolution of role\hyp{}relationship negotiation strategies, comparisons of new and veteran \glslink{member}{users'} linguistic choices, quantitative approaches, and attempts to map longitudinal change in \glslink{discourse-semantic}{discourse and semantics} to shifts in \glslink{lexicogrammar}{lexicogrammatical} features.
%todo: robyn suggests making a new sentence to highlight quantitative shortcoming

\subsection{Consumer-centred and computer-mediated healthcare}

Over the past few decades, there has been increasing recognition within mainstream healthcare institutions that consumer satisfaction and overall health outcomes can be improved through the practising of \emph{\gls{consumercentred} medicine} \cite{stewart_effective_1995}. Under this paradigm, clinicians foster collaborative exchanges with those they treat: greater attention is paid to consumers' feelings and beliefs; consumers are actively involved in decision\hyp{}making processes; clinicians establish long\hyp{}term relationships that can be responsive to consumers' prior journeys through the healthcare system \cite{woodward-kron_international_2016}. The \gls{consumercentred} model thus recognises the centrality of communication and interaction to the practice of medicine, necessitating functional linguistic analysis of \glsxtrfull{HC}---that is, research into the relationship between language use in healthcare and more effective healthcare practice. So far, however, healthcare communication research has typically focussed on clinician--consumer interactions within formal settings, such as hospitals and clinics \cite{slade_communicating_2015}. Despite acknowledgement that the consumer journey extends far beyond his\slash her interactions with healthcare professionals \cite{balka_situating_2010,dickerson_cancer_2011}, and despite the increasing prominence of the interactional Web in daily life \cite{hadlington_cognitive_2015}, little has been done to connect \glspl{consumer}' interactions with clinicians to their use of \glsxtrfull{CMC} \glspl{mode} such as \glspl{OSG}.
%todo: ``Big leap here. new sentence and how about something like ...to invest a ... a functional linguistics analysis "
%todo: placement of slade ref?
%todo: the healthcare journey as assemblage of registers

\subsection{Exploiting computer-mediated discourse}

Just as online healthcare spaces may have unique effects on health, so too do they provide researchers with a unique window into the consumer healthcare journey \cite{harvey_disclosures_2012}. First, as noted above, situation factors present in \glspl{OSG} (i.e. the informal, intra\hyp{}consumer Tenor of interactions) lead to texts that construe the consumer journey candidly, in detail, and unadulterated by potential influence of healthcare professionals. At the same time, the medium factors of \glspl{OSG} (i.e. the way the interactions are produced and archived) make viable the use of state\hyp{}of\hyp{}the\hyp{}art computational linguistic methods: because \glspl{OSG} are generally anonymous, large, well\hyp{}structured and metadata\hyp{}rich, they can be automatically transformed into high\hyp{}quality corpus linguistic resources, grammatically annotated, and searched using tools and methods from corpus linguisics (\gls{CL}). 

%The quantified\slash digitised self
% What Jones refers to as \emph{entextualisation} of the self.

Using these emerging computational methods, it is possible to build reliable, quantitative accounts of healthcare consumers' language choices \cite[see][for a recent example]{maclean_forum77:_2015}. In contrast to the collection and manual analysis of face\hyp{}to\hyp{}face, clinician\textendash{}consumer interactions in formal healthcare settings, computational methods are dramatically more scalable, reproducible and time\slash cost effective \cite{yao_impact_2015}. Moreover, as computational methods improve in speed and accuracy, a number of new applications of computational analysis of consumer language use are beginning to emerge. Large, digitised datasets are being used to predict health outcomes, to identify health risks \cite{kim_use_2013,st_louis_can_2012,oleary_twitter_2015}, and to build intervention programs \cite{chen_dissecting_2015}. Currently, however, computational models of healthcare discourse are often more heavily based on metadata features of \gls{CMC} texts, such as timestamps and geotags, with the authors' actual linguistic content remaining under\hyp{}utilised \cite{yesha_method_2015}. The main reason for this is that \gls{NLP} methods for accurately processing healthcare texts are, in many respects, still in their infancy: despite recent advances in parsing and information extraction, automatic extraction of useful information from large quantities of \glspl{consumer}' health talk is far from a solved task. As such, outside of research environments, clinical applications of \gls{NLP} to date have generally been limited to non\hyp{}critical\slash low-stakes tasks such as supplementary data analysis, speech-to-text systems, and the sorting and filtering of texts \cite{maddox_natural_2015,wasfy_enhancing_2015}.
%todo: solved = awkard

%Core fea- tures of patient-centred communication include foregrounding the patient as a person, taking into account the patient’s preferences, contributions, and psy- chosocial aspects, rather than excluding these aspects and focusing only on the biomedical dimension (Roter 2000; Roter et al. 1997) \cite{woodward2016international}

%  inclusion of the patient’s psychosocial perspective, sharing power and responsibility, establishing a therapeutic alliance between doctor and patient, and acknowledging both the doctor and patient as a person (Mead and Bower 2000; Stewart 2001). The patient-centred care model remains the dominant approach informing medical communication curricula and communication guides (for example, Makoul 2001a; Silverman et al. 2005; von Fragstein et al. 2008) and patient-centred communication is considered a desirable skill for medical graduates (de Haes and Bensing 2009). However, several health care communication researchers have pointed out the potential pitfalls and limita- tions of this approach (Beach and Inui 2006; Roter 2000). For example, a doctor- patient relationship that is characterised by high patient power with low physi- cian power can result in a consumerist model of care, whereby patients set the goals and agenda, and the clinician provides information and technical services (Roter 2000). Negotiating patient preferences to achieve mutually agreed deci- sions can present communication challenges for novice doctors.

%Recent developments in \gls{NLP} make it possible to automatically annotate texts with lexicogrammatical information, and to extract meaningful information from the annotated data. Using these emerging computational methods, it is possible to build reliable, quantitative accounts of healthcare consumers' language choices, both in terms of wordings and meanings.  
%manipulate these annotations in complex ways, with sensitivity to grammar, grammatical metaphor, co-reference, and hypers\slash hyponym taxonomies.
% Recent research has sought to use these kinds of advances on health discourse to identify patterns that may be useful in clinical settings.

% what are the benefits of the above strategies?

%At the same time, \glspl{OSG} are also a potential dataset in medical\slash clinical natural language processing, where information is extracted from texts produced within healthcare institutions, such as radiology reports, doctors' notes, and consumers' narratives \cite{uzuner_chronology_2013}. By automatically analysing patterns in such texts, it is possible to discover drug interactions and adverse events, automatically build medical history timelines, and predict resource allocation and treatment outcomes \cite{paul_social_2016,raghavan_medical_2014}. These outcomes, however, are dependent on the development of methods for locating semantically useful information from \glslink{lexicogrammar}{lexicogrammatical} annotations. This is especially complicated when the phenomena of interest are distributed over entire texts, rather than isolated within individual clauses. A further complicating factor is the informal, unconstrained nature of \gls{CMC}, which can affect the accuracy of parsers and the relevance of texts. \textcite{jones_health_2013} reminds us that outside of the hospital and the clinic, talk about healthcare is very often bundled with talk about other topics as well.

% revise this: it came from the start. gruba doesn't like 'practical is that the'

%The novelty and practicality of \glspl{OSG} as datasets has made them a common site of investigation within functional linguistic traditions \cite[e.g.][]{blank_differences_2010,stommel_online_2010,sugimoto_support_2013}. Novel to many researchers is the ability to access huge repositories of intra-consumer language; practical is that the data come to the researcher in a well\hyp{}structured, digital, searchable form. Lacking thus far, however, have been discourse-oriented studies that take advantage of the immense size and inherent structure of many \glspl{OSG}: though many communities contain millions of words of publicly accessible data, coupled with metadata (embedded information such as timestamps, titles, usernames, gender, location, etc.), most discourse analytic research has involved qualitative analysis of small selections of threads or posts \cite[e.g][]{horne_doing_2009,stommel_conversation_2008,varga2014grieving}.


\subsection{Corpus linguistics and discourse}

% glossaries hacked because it did something unpredictable here
\glslink{CL}{Corpus linguistics} (\glsxtrshort{CL}), and more recently, \glsxtrfull{CADS}, provide both a potential methodological orientation and a set of practices that may assist in quantitative investigation of a \gls{corpus} of \glslink{post}{contributions} to an \gls{OSG}. As a branch of discourse analysis, \gls{CADS} is well\hyp{}suited to use within \gls{consumercentred} healthcare research: both prioritise meaning\hyp{}making and lived experience; both are sensitive to grammar, narrative and context \cite{crawford_language_2013,partington_corpora_2004}. A key strength of \gls{CADS} is its ability to locate what \textcite{gee_social_2007} has called \emph{Big D Discourses} (those concerning social status, power, etc.) within sets of related texts whose topics may be \emph{small d discourses} \cite[about particular things and events in the world; see][]{baker_corpora_2013}. At the same time, the use of \gls{CL} methods in discourse analysis can be used to limit researcher bias and enhance reproducibility: frequency information can demonstrate that the linguistic patterns being discussed are typical of, rather than `cherry picked' from, the very large samples of language from which \glspl{corpus} are constructed \cite{baker_acceptable_2012}. To date, however, \gls{CADS} has focussed more on corpora of news articles, policy documents and government communications than \gls{CMC}. Of the few studies of \gls{CMC} \glspl{corpus} \cite[e.g.][]{harvey_am_2007,harvey_disclosures_2012,prentice_using_2010}, none has dealt with a single, self\hyp{}contained online community or \gls{OSG}, despite the relative ease of building large, well\hyp{}structured \glspl{corpus} from \glspl{post} to a publicly accessible \gls{forum}. Finally, \gls{CL} and \gls{CADS} practitioners generally do not take advantage of state\hyp{}of\hyp{}the\hyp{}art \gls{NLP} tools for annotating and extracting features from natural language \cite{groom_corpora_2015}, limiting the extent to which grammatical patterns can be analysed alongside lexis. This leaves a gap between what is automatically quantifiable (i.e. lexis, and the adjacency of lexical items) and the linguistic strata of interest (meaning, discourse and semantics), limiting the extent to which \gls{CADS} can be automated, and ultimately, the extent to which \gls{CADS} research can inform clinical practice or other non\hyp{}research settings.

%odo: robyn suggests: patterns of consumer engagement with health information

\subsection{Functional linguistics and healthcare discourse}

Central to useful analysis of discursive patterns \emph{en masse} are a conceptualisation of language as a functional resource, an understanding of how words and wordings relate to meaning and context, and an awareness of how structure unfolds through text. The majority of quantitatively oriented studies of online health discourse, however, simplify what language is and how it works: language is often reduced to lexis and collocation of lexis; texts become bags\hyp{}of\hyp{}words \cite[e.g.][]{maclean_forum77:_2015,yesha_method_2015}. Few have explicitly drawn upon functional grammars or theories of language designed to connect linguistic systems and strata in a reliable way.

Perhaps the most comprehensively articulated of contemporary functional grammars \cite{eggins_analysing_2004} is \gls{SFL}, a socio\hyp{}semantic theory of text and context \cite{halliday_introduction:_2004}. \gls{SFL} theorists argue that language is structured in order to achieve three kinds of meaning: \textbf{interpersonal meanings}, which construct and negotiate role\hyp{}relationships between speakers; \textbf{experiential meanings}, which communicate doings and happenings in the world; and \textbf{textual meanings}, which reflexively organise language into coherent, meaningful sequences. These \gls{discourse-semantic} functions are realised by different parts of a language's lexis and grammar. In English, interpersonal meanings are made via the \sctext{Mood} and \sctext{Modality} systems. Experiential meanings are made through the \sctext{Transitivity} system. Textual meanings are made through referential and conjunctive functions, as well as the system of \sctext{Theme and Rheme}. In the context of \glspl{OSG}, the distinction between interpersonal, experiential and textual meanings potentially allows the researcher to separate analyses of social support (via \sctext{Mood} choices) and health information provision (via \sctext{Transitivity} choices), while remaining sensitive to how the architecture of the \gls{forum} itself may affect choices of \gls{THEME}. Despite this potential, and despite recent use of \gls{SFL} in healthcare communication \cite{matthiessen_applying_2013,slade_communicating_2015,woodward-kron_international_2016}, \gls{CMC} \cite{lander_building_2014,zappavigna_enacting_2013} and \gls{CL} \cite{hunston_systemic_2013,thompson_system_2014} contexts, \gls{SFL} has yet to be operationalised within a linguistic exploration of an \gls{OSG}.

%\subsection{Extracting information from online health discourse}

%In \glspl{OSG}, like clinical settings, language is accompanied by rich kinds of metadata concerning speaker demographics, post timestamps and geotags.  Medical research has aimed to use large, digitised datasets to predict health outcomes, to identify health risk, and to build intervention programs. While much attention has been paid to extraction of information from metadata for these purposes, natural language of consumers remains underutilised.

%this paper presents a corpus\hyp{}based investigation of shifting \glslink{lexicogrammar}{lexicogrammatical} and \gls{discourse-semantic} choices over the course of membership in a an online bipolar disorder community. 8.2 million words in 60,000 posts are transformed into a metadata-rich, grammatically annotated corpus and investigated from a systemic\hyp{}functional perspective: member roles are investigated through an analysis of mood and modality choices of new and veteran members' posts. Shifts toward normative strategies for ascribing and attributing bipolar disorder are uncovered by analysis of relational processes within the Transitivity system. Discussion centres on the affordances and limitations of corpus linguistics and \gls{SFL} as strategies for providing quantitative support for key claims of earlier \gls{OSG} research.

\section{Statement of the problem}

%Though increasing attention is being paid to \emph{discursive socialisation} within the forums---that is, the ways in which new members learn and adopt group norms across various strata of language---due to the relative infancy of the research area, currently, potentially useful methodologies from other areas of applied linguistics have yet to be operationalised. 

%\glspl{OSG} are a rich data source that can be used to understand consumer health discourse and the consumer journey, and that may ultimately be used to improve treatment outcomes in clinical settings. Though there have been a number of studies of discourse in \glspl{OSG}, such studies are hampered by two major methodological limitations. First, few researchers have attempted quantitative analysis of language use, limiting the ability to claim representativeness, generalise findings, or apply developed methods to novel communities. This shortcoming is in part caused by a dearth of dedicated computational\slash corpus linguistic tools for interrogating structured and grammatically parsed corpora. Second, though it has been commonly noted that \glspl{OSG} exist to provide both social support and health information, the interaction between these two commodities, as well as the ways in which their \glslink{lexicogrammar}{lexicogrammatical} realisations may change throughout membership, remain poorly understood. Though \gls{SFL} provides a potential framework for this task, to date, the theory has not been applied to an \gls{OSG}.

%The richer understanding provided by a corpus\hyp{}based, theoretically grounded investigation of language use in an \gls{OSG}

%connection between online and offline components of the consumer journey, and ultimately, of how \gls{CMC} data may be useful in the practice of \gls{consumercentred} medicine.

Recent developments across the intersecting fields outlined in the sections above have made it possible to use online health discourse, combined with computational and\slash or discourse\hyp{}analytic methods, to gain new insights into consumers' healthcare journeys and experiences. Discourse-analytic research has already demonstrated how \gls{OSG} \glspl{member} negotiate role\hyp{}relationships, reinforce community values, and give and receive health information and social support. At the same time, quantitative and computational approaches are beginning to show that largely automated methods can also yield insights into the same kinds of data, many of which may be useful in clinical settings. So far, however, despite their overlapping goals, the qualitative\slash discoure\hyp{}analytic and quantitative\slash computational streams of research have yet to be brought together, and as such, each has been unable to profit from theoretical and methodological advances made within the other.

%todo: in para below, robyn wants justification for economic stuff
Because of a lack of dialogue between the two main approaches to \gls{OSG} research, both have identifiable current limitations. The discoure\hyp{}analytic stream, while providing rich insights into social support and information provision online, has yet to be able to describe the interaction between these semiotic commodities, or the ways in which their \glslink{lexicogrammar}{lexicogrammatical} realisations may change throughout membership. Due to this stream's reliance on researcher interpretation of data, results are more expensive to generate, and more difficult to reproduce. At the same time, potential approaches for computationally analysing \glspl{OSG} require further development: computational linguists interested in classification of and prediction from health\hyp{}oriented talk have not taken advantage of existing accounts of how users interact and construe the world online, nor of functional frameworks for connecting instantiated words and wordings to the more abstract plane of meaning more generally. As a result, computational analyses of \glspl{OSG} have prioritised lexis at the expense of grammar, and thus failed to account for the central role of grammar in the meaning\hyp{}making process. % and therefore sacrificed explanatory power

% got to change dominance
The final key problem addressed by the thesis is the narrow focus of existing \gls{HC} literature. Within this research area, despite widespread adoption of the \gls{consumercentred} paradigm, the consumer journey typically only becomes an object of study when the consumer first engages with a health professional, and often ceases to be studied when the consumer leaves the hospital or clinic. This leaves the vast quantities of readily accessible health\hyp{}oriented \gls{CMC} uncharted, and thus, unable to be exploited for the purposes of informing practice or improving health outcomes. Potential tools and methods for extracting insights from intra\hyp{}consumer health discourse, as well as the viability of the endeavour more generally, remain under\hyp{}explored. Moreover, focussing on communication in formal healthcare institutions means that the journeys of those who suffer from health problems, but do not seek treatment through formal channels, are undocumented.

\section{Aims of the thesis}

%todo: robyn wants this in the past tense
The primary aim of this thesis is to investigate linguistic change over the course of membership in an \gls{OSG}. To achieve this aim, I conducted a case study of a large online \glslink{bipolar}{Bipolar Disorder} support group (henceforth, the \emph{\glslink{Forum}{Forum\slash Bipolar Forum}}). \Glspl{post} to the \gls{Forum} were transformed into an annotated, metadata\hyp{}rich \gls{corpus}, structured to contain ten subcorpora of \glspl{member}' \glspl{post} at different stages of membership. These subcorpora were then interrogated for a combination of lexical and grammatical features, in order to locate components of \gls{lexicogrammar} that are \emph{at risk}---that is, likely to vary in frequency---over the course of membership. \sctext{Mood} and \sctext{Transitivity} systems were queried separately, with concordancing used where necessary to build an account of how the located \glslink{lexicogrammar}{lexicogrammatical} phenomena work to make interpersonal and experiential meanings. Because current corpus analysis tools are insufficient for these aims (see Chapter \ref{chap:researchdesign}), a secondary aim of the research project was the development and application of reusable corpus linguistic tools and methods for extracting functional linguistic information from parsed, structured corpora. Accordingly, Chapter \ref{chap:researchdesign} and Appendix \ref{appendix:corpkit} describe an open\hyp{}source Python module, \texttt{corpkit}, which facilitates the construction and functional linguistic analysis of \glspl{corpus} (\texttt{\url{https://github.com/interrogator/corpkit}}). The code used to generate all findings is also available within interactive \emph{Jupyter Notebooks} (via \texttt{\url{https://github.com/interrogator/thesis}}), allowing other researchers to manipulate the dataset and reproduce results.

\section{Scope of the thesis}

Necessarily, the scope of the investigation must be constrained in a number of ways. First is in the selection of a \gls{forum}\hyp{}based \gls{OSG} as the focus of the case study, rather than any of a diverse range of other online \glspl{mode} for health information\slash social support transmission (blogs, chat sites, YouTube videos, etc.). \Glspl{forum} were chosen due to their ubiquity, their sustained history of academic inquiry \cite{kim_process_2012,sillence_why_2013}, and the publicly accessible nature and the comparative ease of harvesting \glspl{post} and their metadata. As such, given that the case study concerns only one community, I can make no definitive claim that the findings are generalisable to \glspl{OSG} for other health conditions, \glspl{OSG} hosted on other domains, or \glspl{OSG} or online communities more generally.
%Another key constraint is the focus on discourses and genres and longitudinal language change within the forum. The same data could foreseeably be useful for research into any number of applied linguistic domains, such as language and identity, illness narratives or stigma. Discourse and genre are of key interest because they are phenomena under investigation within each of the three approaches used in the thesis: much CMDA and \gls{SFL} research has been dedicated to discourse and genre \cite{herring_discourse_2011,martin_analysing_1997}, and both are emerging as strong foci within CL, as evidenced by the growing body of research within \emph{corpus assisted discourse studies} (\gls{CADS}) \cite{baker_acceptable_2012}.
Second, in terms of linguistic areas of interest, the case study analysis centres on \sctext{Mood} and \sctext{Transitivity} choices, as defined by the \glslink{SFG}{systemic-functional grammar} \cite[as in][]{halliday_introduction_2004}. A notable constraint is that the dimension of \gls{Mode}---of \sctext{Theme and Rheme}, as well as reference and conjunction---is largely ignored due to spatial considerations. It must also be noted that although a number of linguistic theories could be useful in an investigation of an \gls{OSG} (e.g. Conversation Analysis, Interactional Sociolinguistics, Frame Semantics), and though multiple theories can be usefully applied simultaneously \cite{eggins_analysing_2004}, \gls{SFL} is the dominant theory used in the thesis, mainly because of its comprehensive treatment of \gls{lexicogrammar}, its usefulness for discourse analysis \cite{widdowson_limitations_2000}, and, to a lesser extent, its history of use in computational contexts \cite[see][]{odonnell_sfl_2005}. Computationally, though tools were developed for extracting features from parsed and structured \glspl{corpus}, and for mapping constituency and dependency annotations to systemic\hyp{}functional constructs, scope did not permit development of tools (or improvement of existing tools) for systemic\hyp{}functional parsing. As a result of these limitations, systemic notions are simplified at times in order to be operationalisable within a predominantly computational workflow.\endnote{Simplification of the SFG for computational purposes, however, has long been the standard approach to computational \gls{SFL} \cite{honnibal_creating_2007,matthiessen_text_1991,costetchi_method_2013}.} 

%todo: needs a copy edit after revision
A further limitation in scope is the ability to perform sustained analyses of individual \glspl{post} and \glspl{thread}. \Gls{forum} \glspl{thread} are dialogic, with meanings made cooperatively between speakers. The main unit of analysis, however, must be the \gls{post}, in order to track how language use in members' contributions changes over time. The genre analysis presented in Chapter \ref{chap:introdata}, while highlighting generic features of language use in the community, is limited to a very small selection of texts, and cannot be taken as representative of the \glslink{Forum}{community} as a whole. Rather, it is intended to be illustrative of the kinds of texts in the \gls{corpus}, and of the difficulty of accounting for genre and context in automated workflows.


% todo: edit paragraph
The final major limitation concerns the kinds of computational tools and techniques applied to the data. While the methods used for the case study analysis often augment and improve currently dominant \glsxtrlong{CL} methods, the research design does not include any of a number of computational linguistic developments that have demonstrated success in classifying texts and linguistic features therein. Techniques such as language modelling and vector space representations, topic modelling, or machine learning approaches to text classification have been shown to outperform rule-based approaches. For the purposes of this case study, however, methods were drawn from those typical of corpus, rather than computational linguistics. In later chapters, I advance an argument that a rigid distinction between the disciplines is unhelpful, and that computational developments could play an important role in expanding the explanatory power of corpus approaches to \glsxtrlong{HC}.

%A final limitation is in the lack of explicit connection of the findings of the case study to concrete clinical outcomes, such as stages of illness or efficacy of treatment. Though the thesis advances an argument that such connections are useful and possible, the methods that can accomplish them are predicated on the existence of either extralinguistic data or human-annotated training data that can be used to automatically classify corpus texts (see Chapter \ref{chap:onlinehealth} for a review of such approaches). Due to the unavailability of these kinds of data (or of resources for producing them), the thesis instead focusses on developing more sophisticated tools and methods for linguistic analysis of health discourse, which may form an important component in future work that aims to use natural language about healthcare for the purposes of predicting outcomes. Accordingly, a research agenda with this goal in mind is sketched in Section \ref{sec:research_agenda}.

\section{Research questions}

\begin{enumerate}\setlength\itemsep{0em}
\item Which components of \gls{lexicogrammar} are \emph{at risk}\slash subject to change over the course of membership within an \glsxtrlong{OSG}?
\item How do these changes relate to \glspl{discourse-semantic} and register in the \gls{OSG}?
\item What implications do the findings generated by this case study have for corpus linguistics, corpus\hyp{}assisted discourse studies, systemic\hyp{}functional linguistics, and healthcare communication research?
\item What kinds of tools and mthods are needed in order to effectively analyse the data, and, more generally, to perform functionally driven analysis of natural language \glspl{corpus}?
\end{enumerate}


%\end{enumerate}
%\section{Data and Participants}
%The site of the investigation is the Bipolar Disorder Forum hosted by Healthboards---one of the more popular of hundreds of groups dedicated to individual conditions. Such forums have been labelled \emph{online support groups}---an online version of face\hyp{}to\hyp{}face support groups. 
%At the time of data collection in December 2013, the \glslink{forum}{board} had received over 66,000 posts and contained almost 9000 threads since its creation in February, 2001. 3588 different usernames have posted at least once to the forum, though it is possible that a small minority of contributors have used multiple accounts. \emph{Lurkers}---those who may read posts, but not contribute, may comprise as much as 90 per cent of the readership \cite{preece_top_2004}.
%Those who have opted to provide voluntary information regarding gender and location reveal that the user-base is approximately 75 per cent female, with proportional amounts of members from developed, English speaking countries (See Appendix A). Contrary to earlier speculation, research has typically revealed that \glslink{member}{users} are unlikely to misrepresent these circumstances \cite{heyd_doing_2013}. Moderators may edit posts that break the stated rules, though moderation is not particularly strict [example?]. Very few health professionals post within the forum. A detailed description of the site is given using terminology from CMDA in the next chapter. Chapter Four describes the transformation of the forum into an annotated linguistic corpus, as well as annotated sub-corpora of first posts to the \glslink{forum}{board} and individual users' post histories.

\section{Research site and approach}

The site of the investigation is the \glslink{bipolar}{Bipolar Disorder} \glslink{forum}{discussion board} of a popular health\hyp{}centred website. The \glslink{Forum}{board} is one of the more popular of hundreds of sub-communities dedicated to individual physical and mental health issues. At the time of data collection in February, 2014, the \glslink{Forum}{board} had received over 66,000 \glspl{post} and contained almost 9000 \glspl{thread} since its creation in February, 2001. \Glspl{post} have been made using 3588 unique usernames. Accounting for the possibility of \glslink{member}{users} creating multiple accounts, well over 3000 unique participants can be safely assumed. \emph{Lurkers}---those who may read \glspl{post}, but not actively contribute, may comprise as much as 90 per cent of the readership \cite{preece_top_2004}.

Corpus\slash computational linguistic tools were used to build a grammatically annotated, 8.2 million word corpus of every post to the \glslink{Forum}{Bipolar Forum}. This \gls{corpus} contains ten subcorpora, reflecting \glspl{post} made at different stages of membership. Using the purpose\hyp{}built tools, the \gls{lexicogrammar} of the corpus was then interrogated for shallow features (e.g. lexical density, part\hyp{}of\hyp{}speech distributions), \sctext{Mood} and \sctext{Modality} features (e.g. Mood and Indicative Type) and \sctext{Transitivity} features (e.g. key Participants and Processes, and the way these typically behave). Using \gls{SFL} as a theoretical framework, \glslink{lexicogrammar}{lexicogrammatical} findings are linked to \gls{discourse-semantic} functions. A brief analysis of generic features of \glspl{member}' first \glslink{post}{contributions} to the \gls{Forum} is also performed, in order to gain a contextually grounded understanding of \gls{Forum} texts, and to illustrate issues in automated \gls{CL} methods. Finally, to ensure that the approach is modelling longitudinal linguistic change, rather than modelling inherent differences between the language use of those who drop out early and those who do not, investigations of alternative corpus structures (where posts from early dropout members are discarded, and where these dropouts' choices are contrasted with those of future veterans) are briefly presented and discussed.

%Following the corpus analysis, the systemic\hyp{}functional conceptualisation of genre \cite{eggins_analysing_2004,christie_genre_2005} is operationalised in a small, qualitative analysis, in order to account for generic influences on contributions to the community.

\section{Contributions of the thesis}

The investigation uncovered a number of \glslink{lexicogrammar}{lexicogrammatical} sites of change within the \glslink{Forum}{Bipolar Forum}. Within the \sctext{Mood} system, though declaratives and interrogatives stayed largely stable in frequency, imperatives and modalised declaratives became increasingly common over the duration of membership, with veteran \glspl{member} explaining to newcomers what course of action they should take. In terms of interpersonal meanings made by these \sctext{Mood} choices, the thesis demonstrates the ways in which member roles and responsibilities within the \glslink{forum}{community} shift over time. Veteran \glspl{member} provide social support and health information to newcomers through advice, blending the speech functions of giving information and demanding goods and services (in this case, actions or general changes in behaviour). This is typified by strategies for advice provision: hypothetical, modalised declaratives (\emph{I would seriously consider changing docs}) become a dominant pattern in veterans' talk with newcomers. These forms foreground lay experience as the source of knowledge underlying claims about what new \glspl{member} should do. In terms of experiential meanings, \emph{metadiscourse}, \emph{vague language} and \emph{jargonisation} emerge as key sites of change over the membership course. \glslink{member}{Users} of the \gls{Forum} are increasingly construed as Agents. Participants and processes associated with instability and negative emotions are displaced by lexis that emphasises stability and control. Shifts can be observed in the kinds of participants and circumstances that attach to key processes, such as the process of diagnosis. Further, the ways in which \gls{forum} \glspl{member} represent the relationship between people and \gls{bipolar} itself change over time: veteran \glspl{member} experientially position \glslink{bipolar}{the disorder} as a possession, rather than an identity---in \glslink{SFL}{systemic\hyp{}functional} terms, as an Attribute, rather than a Value. This distinction foregrounds \glspl{member}' agency over their condition and thus their ability to manage its symptoms and their effect on daily life.

% genre findings, etc
% other corpora

\subsection{Implications for corpus linguistics}

The methodology developed for the investigation presents a number of new possibilities for \gls{CL}\slash \gls{CADS} research. First, I demonstrate the viability of using novel kinds of data and data structures as \glspl{corpus}, showing how corpus\hyp{}based approaches can be used to investigate phenomena such as online communities, role\hyp{}relationship negotiation and socialisation. At the same time, I demonstrate the utility of automatic parsing for discourse\hyp{}analytic work, and provide new tools and methods for traversing these annotations and extracting functionally useful information. Second, I demonstrate the utility of \gls{SFL} as a means of mapping \glslink{lexicogrammar}{lexicogrammatical} change to \gls{discourse-semantic} change, and of delineating interpersonal and experiential metafunctions of language in \glspl{corpus}. Also presented are novel approaches to core \gls{CL} tasks such as relative frequency and keyword calculation, linear\hyp{}regression based sorting, and \gls{corpus} visualisation. Together, these contributions significantly extend the ability of \gls{CL}\slash \gls{CADS} research to both describe and explain novel kinds of texts and to make claims about recurrent discourses through the use of quantitative, corpus\hyp{}based approaches. %  Also presented are strategies for accounting for incongruence, rank shift and grammatical metaphor \cite[as defined by][]{halliday_introduction_2004} during corpus interrogation and analysis. 

\subsection{Implications for systemic functional linguistics}

The case study also has implications for \gls{SFL}, contributing to both theory and the range of registers that have received sustained treatment within the tradition. At the level of theory, the interrelatedness of metafunctions within the corpus data is described. In \gls{SFL}, experiential meanings are not considered to play important roles in role\hyp{}relationship negotiation. In the community, however, experiential choices---such as the distinction between \emph{being} and \emph{having} bipolar disorder---appear to not only communicate propositional information, but also construct and negotiate the newcomer\slash veteran identities within the \glslink{forum}{board}. This tension is also apparent in the way jargon terms are instantiated by veteran members, with jargon appearing to play important roles within both interpersonal and experiential meaning. Interpersonally, jargon demonstrates familiarity with community norms and expectation---solidarity and contact, within the Appraisal system \cite{martin_language_2005}. Experientially, jargon in this context may also be a useful marker of knowledge about health, as jargon typically achieves more delicate distinctions between important participants in a given Field of discourse: \glslink{Forum}{community} \glslink{member}{users} distinguish between kinds of \gls{bipolar} and kinds of anti\hyp{}depressants, for example, with a great deal more specificity than would be expected outside of mental health oriented communities. %Similarly, an analysis of embedded clauses functioning as Verbiage in verbal processes are shown to pattern with Mood and Indicative Type distinctions.

An added contribution of the thesis for \gls{SFL} is its answering of calls for further research integrating \gls{SFL} and corpus linguistics \cite{hunston_systemic_2013,thompson_system_2014}, and in using \gls{SFL} for \gls{consumercentred} healthcare research \cite{matthiessen_applying_2013,thompson2001interview}. The case study also contributes to Matthiessen's call for \emph{registerial cartography}, which aims to situate described registers within maps drawn along either the axis of instantiation or the axis of stratification \cite{matthiessen_modeling_2015}. The articulation of methods and findings presents new ways of doing \gls{SFL} for potential take\hyp{}up by other researchers, and furthers claims that \gls{SFL} can be usefully applied in diverse contexts as a means of describing and explaining the role of language in interaction. The freely available software developed for the case study also integrates a number of systemic functional concepts (insofar as is possible with current parsers), including Process Type matching and translation of Universal Dependencies \cite[see][]{nivre_towards_2015} to systemic labels.

\subsection{Implications for healthcare communication research}

The final implications of the thesis are for \gls{HC} research. The shift toward \gls{consumercentred} healthcare has resulted in a need to account for language use in the diverse kinds of situations encountered by consumers on their progression through formal healthcare systems \cite{matthiessen_applying_2013,slade_emergency_2008}. As \textcite{jones_health_2013} reminds us, however, consumer healthcare journeys exist both inside and outside of hospitals and clinics: consumers' knowledge of health problems, their feelings, and their decision making practices are influenced not only by health professionals, but by friends and family, online and offline. Importantly, unlike \gls{HC} research within clinics and hospitals, interactions within a publicly accessible \gls{OSG} include the voices and lived experiences of those who may be marginalised, and\slash or those who have never sought professional treatment \cite{harvey_disclosures_2012,mautner_time_2005}. The thesis therefore provides an empirical perspective on health discourse of consumers of formal institutions, highlighting their journeys through an emphasis on longitudinal change over the membership course. At the same time, with recognition within medical institutions that \gls{consumercentred} healthcare can lead to better health outcomes \cite{woodward-kron_international_2016}, and that corpus\slash computational methods can be used to extract meaningful information from health discourse \cite{mayfield_automating_2014}, the creation of a reproducible, resource\hyp{}effective methodology for analysing online healthcare discourse facilitates future research into other online communities, health conditions or consumer demographics. As mentioned above, future work based on the methods presented here could also be used to better link \gls{HC} research to clinical outcomes.
%todo: change robyn reference to something it cites
%todo: explain clinical outcomes

%The thesis sketches a research agenda centred on the use of functional, corpus and computational linguistic methods for advancing \gls{HC} research, and a potential workflow that exploits state\hyp{}of\hyp{}the\hyp{}art tools and rich linguistic theory and semi\hyp{}automatically analyses \gls{HC} data, allowing the relationship between healthcare consumers' linguistic practices and health outcomes to be more accurately measured.

\section{Overview of the thesis}

\noindent\textbf{Chapter \ref{chap:intro}} introduces the areas of research, the research questions, and the site of the investigation. It also summarises the contributions of the thesis for \gls{CADS}, \gls{SFL} and \gls{HC}.

\noindent\textbf{Chapter \ref{chap:onlinehealth}} reviews existing linguistic research into online health communities, in order to synthesise and critically reflect on what is currently known about language use in \glspl{OSG}.

\noindent\textbf{Chapter \ref{chap:approaches}} proposes \gls{CL} as a set of methods for investigating \glspl{OSG}, and \gls{SFL} as a theory of language amenable to corpus linguistic analysis of \gls{CMC}. Shortcomings in corpus linguistic tools and methods for doing discourse analytic work are highlighted, as are difficulties in implementing systemic\hyp{}functional concepts into corpus\slash computational linguistic workflows.

%todo: weird too much passive
\noindent\textbf{Chapter \ref{chap:researchdesign}} describes the case study of the thesis. A general description of the \glslink{Forum}{Bipolar Forum} is provided, as well as ethical considerations, and processes for transforming the \gls{Forum} into a linguistic corpus are explained. Tools developed for interrogating the data are described and justified. The approach to analysis is also outlined.

\noindent\textbf{Chapter \ref{chap:introdata}} presents an introduction to the investigation. This involves a bottom\hyp{}up, qualitative, genre\hyp{}based analysis of a small selection of \glslink{post}{contributions} to the \gls{Forum}. The genre analysis is followed by an account of shallow linguistic features in the \gls{Forum}, such as word length, clause and sentence length, and lexical density.

\noindent\textbf{Chapter \ref{chap:interpersonal}} presents findings from the investigation of \sctext{Mood} and \sctext{Modality} features of the \gls{corpus} highlighting their relationship to \gls{Forum} \glslink{member}{users}' interpersonal meaning\hyp{}making practices.

\noindent\textbf{Chapter \ref{chap:experiential}} presents an analysis of \sctext{Transitivity} choices, mapping longitudinal change to shifting experiential semantics.

\noindent\textbf{Chapter \ref{chap:discuss-bp}} addresses Research Questions 1 and 2 by providing a semantically organised discussion of \glslink{lexicogrammar}{lexicogrammatical} and \gls{discourse-semantic} features at risk over the course of membership within the \gls{Forum}. Findings are related to those of earlier \gls{OSG} literature reviewed in Chapter \ref{chap:onlinehealth}. A critical reflection on theoretical and methodological issues of the case study design is also presented.

\noindent\textbf{Chapter \ref{chap:implications}} outlines the implications of the study for \glslink{CL}{corpus linguistics}\slash \glslink{CADS}{corpus\hyp{}assisted discourse studies}, \gls{SFL} and \glslink{HC}{healthcare communication} research. It then addresses Research Question 4, concerning the development of new tools and methods for \gls{CL}.

\noindent\textbf{Chapter \ref{chap:conclusion}} provides a brief discussion of possible future research, followed by a summary of the thesis and a conclusion.
