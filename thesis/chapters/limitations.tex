%!TEX root = ../thesis.tex
%Research agenda

\chapter{Critical reflections on the investigation} \label{chap:reflection}

In the previous two chapters, I presented an analysis of the \glslink{Forum}{Bipolar Forum}, and linked findings to online support group literature. In this part, the focus shifts toward the shortcomings and implications of the case study. This chapter reflects critically on the case study design, including site selection, tools, and analytical approach. The next chapter maps out future directions based on what has been learned here.

%\subsection{Socialisation and linguistic change}

%socialisation, though a possible explanation for these changes, should not be taken as a given. \gls{OSG} research needs to draw a clearer line between longitudinal change and socialisation, with studies of the latter ideally also involving interviews, access to private messages, and the like.

\section{Chapter summary}

In this chapter, I have critically reflected on the methodology and case study, noting limitations that result from shortcomings in available tools and resources, from linguistic phenomena that were unexplored due to time and\slash or resource constraints, and from theoretical gaps. The next chapter takes these limitations into account, in order to outline the implications of the research, as well as the possible future work that methods and findings of this case study have made possible.

%\bibliography{../references/libwin.bib}
