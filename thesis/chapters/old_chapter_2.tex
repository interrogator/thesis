%!TEX root = ../thesis.tex
\chapter{Member roles and identities in online support groups}

   The previous chapter contextualised and outlined the investigations of the thesis. This chapter synergises and critically reflects on research relevant to discourse socialisation in online communities. After providing an account of socialisation in offline and academic contexts, I discuss studies focussed on legitimation and socialisation in online communities. Key shortcomings within the research area are then noted.

   %\textcite{postmes_formation_2000} argue that social identity issues have a stronger presence in CMC environments, due to the deindividuation that occurs by a lack of physical co-presence.

   %For group members ... the properties of the group and behaviour within it (the formation of social norms of conduct and of social identity itself) need to be inferred from others' and one's own actions and the responses to them \cite[p.~345]{postmes_formation_2000}

   %The SIDE model argues that mediated groups can be very real to their members psychologically.

   \section{Socialisation}

	  Broadly, \emph{socialisation}\endnote{There are disputes at the terminological level: \emph{socialisation, acculturation, enculturation, induction, initiation} and \emph{inculcation} have been used both synonymously and contrastively in socialisation literature \cite{duff_language_2010}. Following \textcite{duff_language_2010}, they are considered more or less interchangeable here, with the first preferred.}~
	  concerns the ways in which newcomers or novices learn through participation in meaningful social interaction with more experienced members of groups \cite{ochs_socialization_1991}. \emph{Groups} is a term left strategically undefined: socialisation research may concern groups of any size and type, from small families (where children may learn family values from parents), to communities of practice (where a new employee learns terminology from more senior employees), to entire language groups (where EAL students visit an Anglophone country) \cite{schieffelin_language_1986}. The broad scope of socialisation thus renders it a phenomenon of interest within a broad range of social sciences, informing research in fields such as psychology, anthropology, sociology and applied linguistics \cite[p.~172]{duff_language_2010}. 

	  Within applied linguistics, research has shown that socialisation may take place at any stratum (or \emph{rank}) of language. At micro-level, we may be socialised phonologically, as shown in Polat's \citeyear{polat_nature_2011} study of Kurdish students' acquisition of Turkish accents. Socialisation at the lexical level has been studied predominantly in occupational settings \cite[e.g.][]{wolf_learning_1989}, where jargon is transmitted from older to newer employees. At meso-level linguistic strata, parents have been observed socialising their children to linguistically perform indicators of emotional affect \cite{clancy_socialization_1999}. At macro-levels, recent research has highlighted `political socialisation' in youth groups \cite{lee_processes_2013} and `ideological socialisation' in medical students \cite{harter_exploring_2001}. As \textcite[p.~172]{duff_language_2010} explains, even socialisation targeted at micro-levels simultaneously involves a broader cultural dimension: through interaction with others, learners must necessarily be exposed to meanings about ``normative, appropriate uses of the language, and of the worldviews, ideologies, values, and identities of community members''.
	  
	  The Vygotskian origins of socialisation in child language acquisition theory have led to a strong focus in linguistics on socialisation in (both first and additional) language learning contexts \cite{ochs_socialization_1991}. Due to the relative infancy of computer-mediated communication, the bulk of such research deals with offline contexts (though online research has now become a well-established trend). Students' socialisation to academic norms has been another common interest, as discursive norms between everyday life and graduate study are commonly seen as extremely contrastive \cite{beckett_students_2010}. Such studies have highlighted socialisation toward specific elements of both academic register (nominalisation, passivisation, use of relational verbal groups, essay sub-genres, etc.) and metadiscourse \cite{mauranen__2003}.

	  Academic discourse is also commonly investigated in online contexts. \textciteuthor{beckett_students_2010} used posts to academic discussion boards alongside interviews and surveys to investigate the socialisation of early graduate students to ``graduate school language and culture by old-timer `expert' second-year master's and doctoral students and their professors'' \citeyear[p.~319]{beckett_students_2010}. Though the central concern of the study was students' perceptions of online learning, a key finding related to \emph{how} they learned was that first year graduate students tended to relate theory to personal experiences and anecdotes, while more experienced students were more focussed on the content itself.

		 ~\\\todo[inline,color=green!40]{Here I want to briefly summarise an authoritative academic literacy development text, as literacy and socialisation are related (see below)\\ Belcher, D. (1994). The apprenticeship approach to advanced academic literacy: Graduate students and their mentors. English for Specific Purposes, 13(1), 23–34.\\ Geisler, C. (2013). Academic literacy and the nature of expertise: Reading, writing, and knowing in academic philosophy. Routledge..}

	  \textcite[p.~171]{duff_language_2010} notes that the common distinction between academic literacy and academic discourse socialisation may be worthy of reconsideration: both concepts, she explains, ``are concerned with learning processes, with macro and micro contexts for language development, forms of knowledge and practice valued, material products or tools involved in literacy, and outcomes''. Furthermore, she reminds us that socialisation is not necessarily a process involving ``mindless, passive conditioning''---it is \emph{bidirectional}, in that novices may socialise experts, and is also often contested in some manner by the socialised.

	  It must be noted, however, that the usefulness of academic discourse socialisation studies as theoretical bases for research into socialisation in online communities may be limited, since the scope and substance of the respective communities differs considerably: in comparison to the academic community, access to online communities is low-threshold, norms are far more locally defined, and all members of the community have the ability to interact with one another \cite{postmes_formation_2000,stommel_online_2010}. %A further issue is that in the case of academic discourse, the definition of norms is a collective task shared by sub-communities of academics. This is in contrast to online support groups, where norms are often defined and maintained the totality of the members of a the individual community.

   \section{Online communities}

	  Online communities---that is, ``mediated social spaces in the digital environment that allow groups to form and be sustained primarily through ongoing virtual communication processes'' \cite[p.~986]{shen_effects_2013}---are of increasing interest to socialisation researchers for two key reasons. First is their growing popularity, which both increases the amount of communities available for research and the amount of people who may be affected by online community research. Second is their practicality as data-sources: access to innumerable communities is round-the-clock and global; recording and transcription are unnecessary; and in some cases a great deal of demographic data is readily available \cite{leech_new_2006}.

	  In general, empirical studies have shown that when contrasted with face-to-face equivalents, online communities have fewer barriers to entry, higher dropout rates and a comparatively high proportion of peripheral membership \cite{zhang_peripheral_2001}. That said, the efficacy of comparisons between face-to-face and computer-mediated communities has recently been questioned, as daily life becomes increasingly mediated by and merged with digital technologies \cite{wu_is_2013}.  %Misconceptions about CMC/D: enregisterment, metadiscourse of difference, though many of these features are relatively rare \cite{squires_enregistering_2010}

	  Applied linguistics is well represented within online-community-focussed research, largely due to the central role of language within many online communities. It has long been argued  by researchers that the limited semiotic resources available in many online environments (chiefly, a lack of physical co-presence) renders language the most suitable resource for identity construction and role relationship negotiation \cite{thorne_computer-mediated_2008}. As \textcite[p.~303]{lam_language_2008} notes, ``language practices are instrumental in creating the norms of behavior of particular online groups and how these norms function to provide sociability, support, information, and a sense of collective identity''. A similar perspective is provided by \textcite{postmes_formation_2000}, who argues that social identity issues have a stronger presence in CMC environments, due to the deindividuation that occurs in part due to reduced cues online.

	  \subsection{Legitimacy}

		 % Concerning for researchers of online support groups has been the ability for members to linguistically construct an apparent level of expertise or legitimacy that is at odds with their actual health literacy, potentially leading to the provision of poor quality health advice. As explained earlier, this is typically seen as a more pressing concern in text-based, asynchronous CMC, where members have few external semiotic indicators of their status within the group, and are thus incentivised to construct their legitimacy through discourse \textcite{varga_grieving_2013}.

		 As CMC in many cases necessitates linguistic self-construction, researchers have turned their attention to the ways in which nonverbally communicated information may be conveyed through language online. Common to date have been studies of the readily apparent micro-level innovations of CMC, such as emoticons, nonstandard orthography and abbreviations \cite{dresner_functions_2010}. Less common, however, are investigations of the ways in which legitimacy and credibility may be established at the discourse-semantic level.

		 Legitimacy has had two major definitions in applied linguistic literature. First is the way language users construct the appropriateness of a given action or behaviour, such as \textcite[p.~782]{reyes_strategies_2011}, who looks at rhetorical strategies for constructing legitimate action in political talk. Reyes highlights five strategies, each of which is realised through specific elements of the lexicogrammar.

		 ~\\\todo[inline,color=green!40]{Table of five strategies and \glslink{lexicogrammar}{lexicogrammatical} realisations here}

		 These are more specific instantiations of the four major strategies theorised by \textcite[p.~92]{van_leeuwen_legitimation_2007}: \emph{authorisation} (reference to history or authority figures), \emph{moral evaluation} (reference to social values), \emph{rationalisation} (references to goals and hegemonic social action) and \emph{mythopoesis} (narratives casting legitimate action and those who perform them in a favourable light). %stole this ! whoops!

		 The second, related sense of \emph{legitimacy} concerns the ways in which the self is constructed as legitimate in order to increase the perlocutionary force of one's own utterances \cite{austin_how_1975,roberts_communicative_1996}. This may be in turn broken down into two main foci within the literature: first is the way in which new users construct their first contribution; second is the ways in which established members give advice in replies\endnote{To some extent, this division in the literature is an artificial and potentially unhelpful one: Investigation B reveals that nearly half of all first contributions to the bipolar forum are replies to others.}.

		 \subsubsection{New members and their first contributions}

		 The majority of studies within this area have analysed first contributions to online communities, with particular emphasis on the ways in which posts may be designed to demonstrate the fulfilment of membership criteria and elicit useful replies. \textciteuthor{varga_grieving_2013} analysed more than 100 first posts to a grief-oriented OSG with the purpose of uncovering ``how grief is socially constructed through talk'' \citeyear[p.~2]{varga_grieving_2013}. Three mains themes were identified and argued to be strategies intended to elicit responses from members: presentation of an atypical story, uncontrollable emotional state and ``troubles telling''---posts in which the new user explains a problem, but does not explicitly request advice. Though not explicitly informed by a genre-theoretical approach, the authors nonetheless argue that structural patterns underlie the analysed messages:

		 \begin{quote}\small\singlespacing
		  We found that newcomers opened their initial posts with stories that began at the event of loss and then moved to establish the background of their relationship with the deceased. Emphasizing the unusual circumstances of their loss and the depth of their connection with the deceased provided an account for their grief. Newcomers continued their accounts through descriptions of their uncontrollable emotional and physical symptoms, which worked to display affiliation with members of the group and to make the case for their legitimate entry \citeyear[p.~5]{varga_grieving_2013}. 
		 \end{quote}

		 \noindent The authors point out, however, that their purely qualitative approach limits the generalisability of the study, noting that further research is necessary to confirm their exploratory results.

		 \textcite{smithson_membership_2011} describes the legitimising function of first posts to a self-harm forum: new members were observed setting out their credentials for group membership, giving narrative medical histories, utilising medical jargon and making reference to other related communities in which they have participated. This is line with the assertion of \textcite{varga_grieving_2013} that ``story formulations often serve particular functions in discourse, such as displaying affiliation with a group and establishing eligibility for group membership''.

		 Horne and Wiggins' \citeyear[p.~173]{horne_doing_2009} study also deals explicitly with structural components of first posts. The paper explores the difficulty faced by members who present as suicidal, but whose continued presence in the forum serves to render illegitimate their suicidal identity. The study involved corpus linguistics, though the sample size of 329 posts is acknowledged as perhaps being too small for keywording and collocation analysis. The authors identify three major types of messages: \emph{life narratives}, in which medical history is presented without a specific addressee; \emph{immediate threats}, which are typically short, containing present and future tense; and \emph{requests}, which involved asking for advice and the use of mainstream medical terminology. The latter category received the fewest replies. The authors hypothesise that this is due to two factors. First, there is a lack of urgency within the lexicogrammar of \emph{request} posts, leading to an impression that the new user is ``inauthentically suicidal'' \citeyear[p.~180]{horne_doing_2009} . Second, explicit requests for advice construct potential responders as equally inauthentic, as it casts them as using the OSG for purposes other than support seeking.

		 An issue in \textcite{horne_doing_2009} (as well as others, \citeNP<e.g.>{stommel_online_2010}) is that the author treats the presence or absence of replies as indicators of a post's success. While this may be a sensible or convenient assumption when doing larger, quantitative-based studies of number of replies, it is perhaps less reliable in small-scale qualitative research: there is no explicit evidence concerning whether or not replies would have been posted if the first post were written differently. Moreover, there is no reliable way to tell that posts were even seen by those who would be likely to reply.

		 It should also be noted that notion of new membership in OSGs itself is rendered problematic by the phenomenon of lurking: first-time posters may have read the forum posts without posting for years, and are thus intimately familiar with the community's discursive norms without ever having produced language within the register \cite{dennen_pedagogical_2008,han_social_2012,preece_top_2004,smithson_membership_2011}. From a sociocultural approach to linguistic socialisation, as non-active participants, lurkers who choose to post would be hypothesised to struggle to fluently invoke appropriate discursive norms. Empirical observation, however, has shown that such users often draw upon the normative genre and register appropriately \cite{weber_missed_2011}. Such users often flag in their posts the fact that they have lurked for extended periods before posting \cite{weber_missed_2011}, perhaps as a means of explaining the reason for their fluency.

		 \subsubsection{Replies and advice to new contributors}

			A smaller proportion of legitimacy research has concerned the ways in which legitimacy is conveyed in replies. For the most part, these studies address a hypothesised concern that members may take advantage of the lack of social cues in the online community and linguistically construct an apparent level of expertise or legitimacy that is at odds with their actual health literacy \cite{varga_grieving_2013}. Evidence regarding the existence and ramifications of this phenomenon is conflicting, however \cite{sillence_giving_2013}: \textcite{hoch_information_1999} found that six per cent of advice provided in an epilepsy forum was objectively wrong, while Sillence's study of a breast cancer found that ``only a very small amount of messages observed in the present study reflected a lay belief or misbelief in  [patient-controlled analgesia] treatment'' \citeyear[p.~8]{sillence_communicating_2012}. \textcite{smithson_problem_2011} found that replies to new members of a self-harm forum were surprisingly mundane, with ``go and visit the GP'' being by far the most common advice dispensed, perhaps further ameliorating cause for alarm.

			Either explicitly or implicitly, studies of replies in OSGs have centred on the notion of \emph{advice}. Defined here as ``opinions or counsel given by people who perceive themselves as knowledgeable, and/or who the advice seeker may think are credible, trustworthy and reliable'' \cite[p.~519]{decapua_strategies_1993}, advice is interesting due to its inherent ties to legitimacy and its combination of interpersonal and experiential purposes. Furthermore, in OSG contexts, health advice is complicated by the complex power dynamics involved in peer-to-peer advice transmission \cite{kouper_pragmatics_2010}. Finally, advice is notable for its potential influence over users' decision marking \cite{sillence_giving_2013}.

			~\\\todo[inline,color=green!40]{A larger treatment of advice as combining interpersonal and experiential components here, including: \\ Locher, M. A., \& Hoffmann, S. (2006). The emergence of the identity of a fictional expert advice-giver in an American Internet advice column \\ Sillence, E. (2013). Giving and receiving peer advice in an online breast cancer support group. \\ Kouper, I. (2010). The pragmatics of peer advice in a LiveJournal community. Language at Internet, 7, 1.}

		 \subsubsection{The lay-expert/proto-professional}

			In medical contexts, a well-noted strategy for legitimation is the instantiation of a \emph{lay-expert} or \emph{proto-professional} register, which is realised at both \glslink{lexicogrammar}{lexicogrammatical} and discourse-semantic levels. Lexically, it may involve the use or appropriation of medical jargon \cite{harvey_disclosures_2012,sullivan_gendered_2003}; discursively, the lay-expert has been shown to foreground legitimate personal experience and display emotional affect. % and on the other hand ...

			\textcite{thompson_credibility_2012} present a series of interviews with members of \emph{patient and public involvement panels}---laypeople consulted by medical institutions in the UK in an attempt to involve the public in the medical research process. They noted that those in lay-expert positions may champion their lack of formal medical training or employment, arguing that it affords a unique point of view that is uncorrupted by financial or socio-political factors. Simultaneously, a high degree of deference to the opinions of health professionals was also observed: interviewed participants `not only supported the dominant techno-scientific discourse around research, but also seemed to defer readily to it in place of their own experiential expertise' \citeyear[p.~609]{thompson_credibility_2012}. That said, it should be borne in mind that this finding may be strongly influenced by the social context in which the interviews took place. As the participants of the study were afforded privileges and treated as having `honorary' roles by health professionals, their construction of both their unique perspective and health professionals' ultimate superiority may be influenced by the desire to maintain their current position and role. Indeed, studies of OSGs with ideologies opposing those of mainstream medicine have noted that hospitals, health professionals and diagnoses are often cast in a negative light and treated sceptically \cite{mulveen_interpretative_2006}. Related is the already-noted finding in \textcite{horne_doing_2009} that posts aligned with mainstream opinion and featuring information about doctors' diagnoses were ignored by other members.

			%The notion of the lay-expert has been problematised by \cite[p.~53]{prior_belief_2003}, who contends that the term is oxymoronic and unhelpful...

	  \subsection{Discourse socialisation in online communities}

		 As both the meaning and scope of \emph{discourse} is broad and often contested \cite{gee_discourse_2004,gee_introduction_2013}, it is common for researchers to characterise, rather than define the term. For this thesis, it suffices to adopt Martin and Rose's simple characterisation of discourse as being ``more than a sequence of clauses'' and ``more than an incidental manifestation of social activity'': discourse is ``meaning beyond the clause'' and as ``the social as it is constructed through texts'' \citeyear[p.~1]{martin_working_2003}.  %Even with a wide definition, however, few studies have directly concerned discourse socialisation in online communities \cite{lee_new_2014}.

		 In online communities, socialisation at the level of discourse has been investigated from a wide array of theoretical perspectives and methodological approaches. \textcite{lee_new_2014} approach discourse socialisation from a member-life-cycle perspective, arguing that members transition through a number of roles during their time within the online community, and that each role has accompanying needs and responsibilities. Newcomers have strong needs for both information and social support, but may not contribute due to the potential for a loss-of-face if information they provide is judged by experts to be incorrect \cite{fuller_innovation_2007}. At later stages in the member life-cycle, users become less anxious about producing content, but lose the motivation to seek out information or support \textcite{lee_new_2014}. %\textcite{lee_new_2014} is more concerned with the relationship between user-generated content and membership retention 

		 In a similar vein, \textcite{danescu-niculescu-mizil_no_2013}, which focuses on the `user lifecycle' of an online beer enthusiasts' forum using a corpus-based approach. They find that new members are responsible for both the introduction and a large amount of the uptake of new jargon (lexical innovation). The register of already established members is argued to become more conservative or fossilise over time, due to the perception that their communicative competence is already sufficient, as resistance against inbound norms, or as a means of marking their seniority.

		 \textcite{cassell_language_2005} used quantitative analysis, content analysis and interviews with participants in a global political forum to track linguistic change over time. They identified three key changes. First, the plural first person pronoun \emph{we} became more frequent when compared to singular \emph{I}. Second, members gave more feedback on others' opinions, rather than promoting and explaining their own. Third, it became more common to collaborate to pursue shared goals. \textcite{smithson_membership_2011} also address discursive socialisation their analysis of a purpose-built self-harm OSG: in the days immediately following the site's creation, normative practices emerged: all active members were expected to provide others with social support using appropriate degree of emotional affect \cite{smithson_problem_2011}. 

		 \textcite{weber_missed_2011} focuses on the role of dispute and conflict in the socialisation process of new members of an online sexual abuse support group. The author contends that disputes provide a context in which norms can be made explicit by experts. After flouting group norms, a new member is chastised by veterans: as Weber explains, ``since she did not learn by lurking, she has to learn by direct instruction'' \citeyear[p.~1]{weber_missed_2011}. After some contestation, rather than leave the forum, the new user eventually ``apologizes, self-deprecates, claims technical and social ignorance, and highlights her need to learn'' (p.~12). It is argued that the newcomer's radical shift in orientation was the result of a realisation that future participation in the group was contingent on her adoption of the discursive features typical of newcomers.

		 It is also worth noting that \textcite{weber_missed_2011} provides a basic account of the generic structure of newcomers' posts:

		 \begin{quote}\small\singlespacing
		 Contents typically found in newcomers' messages include: a greeting; a description of the person's contact with the group thus far; a reference to sexual abuse experiences or related problems; and a request (p.~4).
		 \end{quote}
		 %
		 Particularly relevant to this thesis is the study of a bipolar disorder forum undertaken by \textcite{vayreda_social_2009}, which uses a conversation analysis approach to explore ``an apparent contradiction between a new user's first post and forum members' replies with ostensibly unsolicited advice'' \citeyear[p.~931]{vayreda_social_2009}. They found that new forum users rarely explicitly asked for advice, but were provided with it nonetheless. The authors contend that advice was thus unsolicited, but complementary: new users made a ``low bid'' by only vague specification of the kinds of replies they sought; established users took this opportunity to direct the user to shift toward group norms \citeyear[p.~940]{vayreda_social_2009}.

	  Furthermore, the study uncovered discursive socialisation toward a biomedical account of bipolar disorder: veteran users of the forum constructed bipolar disorder in the same terms as well-established ICD and DSM guidelines and stressed the necessity of treatment through formal, mainstream channels. In stark contrast to \textcite{horne_doing_2009}, who found that discussion of diagnosis led to being ignored in a suicide forum, \textciteuthor{vayreda_social_2009} demonstrate that diagnosis is a prerequisite for legitimate community membership: even posts displaying an alarming sense of urgency were met with terse commands to ``get an appointment with a psychiatrist'' when the new member did not explicitly mark their status as diagnosed \citeyear[p.~940]{vayreda_social_2009}.

		 The relevance of the authors' conclusion to this investigation is striking:

			   \begin{quote}\small\singlespacing
			   That instruction to go straight to the psychiatrist shows, in microcosm, the ideology of the forum. It crystallizes the site's motivating spirit, visible from its title page, and on which we have already commented: that only one account of bipolar disorder will be countenanced, and that is the biomedical. No time is afforded to any consideration of the user's symptoms or circumstances. For the forum, diagnosis must be in the hands of the psychiatric profession, the ultimate authority on the illness and its treatment; the forum offers support, information and, indeed, advice, only on that basis. ... The structure of open request allows the response to choose its path. And that path is biomedical diagnosis. Once she has that, then she can enter the community of forum users and the site's resources will be at her disposal \citeyear[pp.~940--941]{vayreda_social_2009}.
			   \end{quote}

	\noindent That said, it is at this point perhaps worth questioning the authors' first finding, which was that advice in many cases was unsolicited. At issue is that the authors take a narrow view of what exactly constitutes a request for advice, discounting formulations in which requests may be incongruently realised, as in the following:

	  \begin{quote}\small\singlespacing
	  ``It'd be great if someone could share this first stage of acceptance with me, or tell me how they got through it.''
	   \end{quote} 

	  \noindent Here, though the new user chooses a conditional declarative rather than an modalised interrogative as a means of realising a request, given that requesting advice involves deference and a potential loss of face \cite{brown_politeness:_1987}, such hedging does not seem inappropriate. Aware of the fact that expert members are fluent in the discursive norms, a new user could perhaps expect that such heavily modalised constructions could be unpacked and decoded as requests for advice.

		 %%%%%  %  \cite{sillence_giving_2013} Analysis of message content has revealed that support groups vary according to the types and frequency of their exchanges.


	  %\subsection{Self construction}

		 %Goffman's work forms not only a major theoretical dimension of much language and identity research generally \cite{edwards_language_2009}, but of CMC research as well \cite{birnbaum_taking_2008,hogan_presentation_2010}. Particularly relevant is his dramaturgical metaphor, whereby individual \emph{actors perform} identity for an \emph{audience} who \emph{monitors} the action. Included in this metaphor are the notions of \emph{front} and \emph{back stage}: when actors are `front stage', they perform according to social rules to avoid facing sanctions and losing face \cite{bullingham_presentation_2013}; back stage is `a place, relative to a given performance, where the impression fostered by the performance is knowingly contradicted as a matter of course' \cite[p.~112]{goffman_presentation_1959}.

		 %Social identity theory and self-categorisation theory are also relevant

		 %Constant flux \cite{danescu-niculescu-mizil_no_2013}

		 %Within virtual communities, individuals aremotivated to derive a common social identity with the collective group, so as to achieve a positive self-concept (Dholakia, Bagozzi, & Pearo, 2004). This common social identity is activated and maintained through regular participation within group activities and active social interaction with other members (Ellemers, Kortekaas, & Ouwerkerk, 1999; Postmes, Spears, & Lea, 2000). Participation in virtual groups confers cognitive self-awareness about one’s statuswithin the group (Ashford&Mael, 1989),while also increasing emotional attachment and commitment to the group (Ellemers et al., 1999; Parks&Floyd, 1996) and serving an evaluative purpose influencing one’s collective self-esteem (McKenna & Bargh, 1999). \cite{phua_participating_2013}

		 %\subsubsection{Group norms/collective identity}

			%It is increasingly acknowledged that there is no blanket effect of technology on computer-mediated communication: CMC is more commonly framed as being reciprocally influenced by technological factors and the social context in which interactions take place \cite{postmes_formation_2000}. 

			%In an early study of online group identity, \textcite{postmes_formation_2000} analysed the evolution of a register for emails sent between participants in a free computer-based statistics course. The authorws noted three main things. First content and form online does have a normative influence over CMC users. Second, conformity increases over time. Third, the developed group norms did not transfer outside of the group---students' messages to staff did not show the same technological influence.

			%\textcite{pfeil_social_2011} found that members of an OSG for older people adhered to one of six `roles', with different practices concerning seeking and providing advice and support, size of messages and the amount of overall interaction with the group.
			
			
			%The best methodologies for systematically determining group norms and the social actors able to influence them are still far from fully explored.

			% ~\\\todo[inline,color=green!40]{I plan on adding a summary of a few more papers on online group identity, with particular focus on methodologies currently used to `measure' norms, as this is the focus of Investigation A}

			% group norms \cite{zhou_understanding_2011}
			%\cite{blanchard_model_2011}
			%\cite{postmes_formation_2000} - 3 findings: content and form is normatice,w ith group norms defining communication bpatterns; conformity increases over time; communication outside the group is governed by different norms. results show that norms prescribing a particular use of technology are socially constructed over time at the level of locally defined groups and also show that the influence of these norms is limited to the boundaries of the group. It is concluded that the process of social construction is restrained by social identites that become salient over the course of interaction via CMC.

			%Quantitative studies of group linguistic norms are largely absent from socialisation research. This is surprising, since online commmunities are intuitively the most pragmatic datasource for such research: the fact that data are globally accessible, preserved ocer time, predigitised and often organised with information about the interlocutors, time and location of interaction...

	  %\subsection{New members and their first contributions}

		 %A substantial proportion of research into online communities has involved a deliberate focus on new members and their first contributions to the community. Though a number of theoretical positions have been adoped within these invstigations, common underlying concerns are discursive construction of legitimacy and early socialisation. Literature from each domain is summarised in the sections below.

		 %reichers_interactionist_1987,kraut_dealing_2010,lee_new_2014}

	  %member retention focus, machine learning to code information and emotional support! \cite{wang_stay_2012} - more emotional support patterns with sustained membership!
			%\cite{dove_making_2011} does well to differentiate first posts linguistically, as a genre perhaps, but uses the `success'ful interaction idea ...
			%\cite{weber_missed_2011}
			%The authors used discourse analysis to examine 107 initial posts to one such group to examine how newcomers constructed their initial posts to display their eligibility for membership. The authors identified three discursive features: formulating unusual stories of loss, describing uncontrollable emotional and physical states, and engaging in “trou- bles telling.” \cite{varga_grieving_2013}


			%Increasingly common, given unprecedented access to health information etc
			%\cite{thompson_credibility_2012} features of lay expert and credibility

			%\cite{thompson_credibility_2012} explain that their interviewees, upon asked what they can bring to cancer research, tended to emphasise the usefulness of having an experienced patient---a `real' person, with `extra-scientific' knowledge. Contasts with clinicians and researchers were drawn to cast lay-expertise in a positive light. INVARIABLY!

			%detached from the financial, bureaucratic, motivated part of the medical world...

			%participants in this study appeared to privilege professional, or ‘certified’, forms of expertise. Moreover, in their role as PPI participants in research settings, it was apparent that they not only supported the dominant technoscientific discourse around research, but also seemed to defer readily to it in place of their own experiential expertise. \cite[p.~609]{thompson_credibility_2012}

			%For example, Shirley was very clear that her PPI role should not involving challenging or questioning the ‘experts’ and she accounted for this in terms of the ‘traditional’, paternalistic model of healthcare interactions:  \cite[p.~610]{thompson_credibility_2012}

			%\cite{kerr_shifting_2007} hybrid position in scientific forum between lay and expert: the experientially honourary type expert

	  %Discourse as one stratum of socialisation: Gee, Lam etc focus broadly on linguistic or language socialisation ...  discourse draws attention to the macro-level features

	  %GMeanings are ultimately rooted in negotiation between different social practices with different interests by people who share or seek to share some common ground. Power plays an important role in these negotiations. ... The negotiations which constitute meaning are limited by values emanating from ``communities''---though we need to realize it can be contentious what constitutes a ``community''”---or from attempts by people to establish and stabilize, perhaps only for here and now, enough common ground to agree on meaning.\cite[p.~6]{gee_social_2007}
	  
	  %\subsection{Effects on social support and information exchange}

	  %Social support, defined as Social support refers to communication between individuals that enhances recipients’ self-esteem, provides stress-related interpersonal aid, andmitigates stressful situations caused by a negative health condition (Kim et al., 2012).

	  %‘verbal and nonverbal behavior that influences how providers and recipients view themselves, their situations, the other, and their relationship and is the principal process through which individuals coordinate their actions in support-seeking and support-giving encounters’ \cite[p.~532]{kim_process_2012} - keep reception and provision in mind as distinct duh

		 %\cite{wang_stay_2012} - more emotional support patterns with sustained membership! - information provision not significant

   \section{Gaps in the research area}

	  Despite an impressive amount of applied linguistic research into socialisation, much is of only limited use for the investigations presented in this thesis. Language acquisition and offline (medical) contexts and academic discourse socialisation comprise a large proportion of research in the area. Less research has explicitly concerned discursive socialisation in online communities or OSGs, and those that have have most often been limited to studies of first posts, despite the fact that the development of the lay-expert/proto-professional register appears to require sustained levels of interaction.

	  An additional issue is that a number of potentially relevant methodologies are yet to be instantiated within the research area. From the literature review, I located three main areas into which the import of well-established methods from other applied linguistics domain may prove beneficial. First, a lack of standardised terminology and methods for the description of online communities hampers cross-applicability of many existing studies. Second, despite the fact that a core motivation for applied linguistic OSG research is the simultaneously exchange of social support and health information, a theory of language that separates these two functions has yet to have been employed. Third, it is unfortunate that many studies of new member interactions have relied on impressionistic and qualitative perspectives, when practices from corpus linguistics allow the building of specialised corpora of first posts for the purposes of quantitative or mixed methods analysis.

	  Each of these issues is briefly expounded in the sections below.

	  \subsection{Site description conventions}

		 A major issue within online discourse socialisation research is that overall descriptions of the site do not adhere to standardised principles and practices, complicating comparability of findings. Terminology likewise is at times contradictorily applied: as just one example, common has been treatment of \emph{mode} and \emph{genre} as synonymous \cite{herring_gender_2006}, despite formal delineation of these concepts as separate within a number of theories of language and context \cite{halliday_introduction:_2004}. Furthermore, potentially relevant information concerning forum demographics is omitted from many studies. While in many computer-mediated contexts, precise figures concerning demographics are difficult to establish, there are nonetheless a number of ways to gather at least indicative demographic data such as user locations and post counts. %Such data may afford insights into the role played by regional linguistic variation or computer literacy.

		 The motivation for a codified system for site classification increases with the knowledge that the number of different modes in which people interact online is constantly increasing. As the ways in which technological factors influence the kinds of communication appearing within the OSGS remain largely unknown \cite{herring_faceted_2007}, transparent classification of features of online environments may prove an important step in the direction of future-proof research.

	  \subsection{Theory of language}

		 The literature review confirmed that many studies undertaken into discourse socialisation are linguistically more or less atheoretical: content and thematic analysis are by far the most common approaches within qualitative research. Of theories of language commonly applied to OSG analysis, conversation analysis has been by far the most common, in large part due to its association with communities of practice research and its strength as a means of understanding turn-taking and topic organisation \cite{stommel_conversation_2008}. Interactional sociolinguistics \cite{hymes_communicative_1972} has been represented to a lesser extent, mostly facilitating language and identity centred discussion.

		 While myriad studies have noted the dual purpose of OSGs as sites for both social support and health information exchange \cite{attard_thematic_2012}, few have attempted to delineate how these two purposes are realised in the \glslink{lexicogrammar}{lexicogrammatical} choices of community members. As will be introduced in the next chapter, systemic functional linguistics provides an exemplary grammar for this task: SFL contends that language users simultaneously attend to interpersonal and experiential goals, and that these two goals are realised with overlapping grammatical systems \cite{halliday_introduction:_2004}. 

		 For the same reason, SFL is of foreseeable benefit to research into advice, which necessarily draws upon interpersonal and experiential meanings. A key issue in \textciteuthor{vayreda_social_2009}'s study of unsolicited advice in a bipolar disorder forum was that \emph{advice} itself was not defined according to \glslink{lexicogrammar}{lexicogrammatical} evidence and and functional parameters. SFL provides a means of ameliorating this concern.

		 Finally, the work on genre within SFL \cite[e.g.][]{eggins_analysing_2004} is yet to be operationalised within online discourse socialisation research, despite the fact that many researchers have highlighted generic structures in new online community members' first contributions. Delineation of genre stages and an overall generic structure would potentially illuminate an important element of discursive norms within online communities.

		 %It is highly likely that the effectiveness of such groups depends on the communications that members exchange with one another, but surprisingly little systematic research has been devoted to specifying how the quality and quantity of such communications affect groups‟ outcomes and members‟ health-related outcomes \cite[p.~1]{wang_stay_2012}

	  \subsection{Data collection practices}

		 The final shortcoming within online discourse socialisation research concerns the fact that methods of data collection and interrogation lag significantly behind what is achieved in adjacent research areas of \emph{web corpus linguistics} and \emph{corpus assisted discourse studies}. \textcite{danescu-niculescu-mizil_no_2013} analysed forum norms quantitatively and chronologically with great success, uncovering the relationship between new members, veterans and jargon terms. The study is predominantly concerned with lexis, however: norms at the level of discourse-semantics are largely unconsidered. Though-corpus assisted discourse studies has shown promise as a means of highlighting the discursive construction of ideologies within large sets of related texts \cite[e.g.][]{koteyko_climate_2013,salama_ideological_2011}, few elements of the approach have been operationalised within socialisation research.

		 %Exploitation of these common features of CMC would forseeably improve CMDA's current guidelines for data sampling. Currently, each noted sampling technique takes place before any data analysis occurs (see Table \ref{tab:sampling}). In this way, a key resource is discarded without ever being drawn upon. Where possible (i.e. when sufficient data exists and can feasibly be collected), it seems sensible to instead involve the aggregated data in the sample selection process, as is often done in CADS: if researchers attempt to first harvest \emph{all} relevant language within the environment, rather than sampling, basic corpus linguistic interrogation could reveal insights that may inform the selection of texts in a systematic, data-driven manner. Keywords and key clusters can be concordanced; concordances exemplifying the quantitatively significant phenomenon can be linked to the contextualised data, which can undergo qualitative analysis. This enhances researchers' ability to claim representativeness of their sample and generalisability of their findings.

		 %As the transformation of large quantities of natural language online into corpus data is easily achievable with techniques considered rudimentary within corpus building, NLP and computer science, this thesis introduces tenets of these approaches---in particular, corpus linguistics---to CMDA. Chapter 4 discusses basic corpus linguistic theory and outlines the successes of corpus linguistic approaches to discourse research within the emerging field of CADS. Chapter 10 establishes guidelines for larger-scale data collection.

   \section{Chapter summary}

	  In this chapter, I outlined empirical research relevant to discourse socialisation in online environments. Shortcomings in current understanding were highlighted.

	  In the next chapter, I review relevant literature concerning computer-mediated discourse analysis, systemic functional linguistics and corpus linguistics in order to develop actionable methodological parameters for the investigations presented in Part II.

%\bibliography{../references/libwin.bib}




















%\subsection{Computer mediated discourse}

%Herring has defined computer-mediated discourse as `the communication produced when human beings interact with one another by transmitting messages via networked computers' \cite[n.p]{herring_computer-mediated_2001}. Thus, CMD very often overlaps with CMC. There are some key differences, however. First is that a single word or sentence sent through an online channel constitutes CMC. As discourse extends beyond the sentence barrier and potentially between multiple participants, a sample must be larger in order to qualify as CMD. Similarly, as discourse constructs and is reflexively constructed by its context, CMD data must be language in context, whereas CMC may still be considered as such when it has been decontextualised (by copy and pasting, for example). Finally, as context in online communication is almost always rendered multimodally (avatars and hyperlinks; font and background colours, etc.), it logically follows that CMD, as contextualised language, must therefore inherently be concerned with multimodality to some extent, while CMC data may consist purely of written text.

%We can surmise from this that DA and CMDA are congruent: DA carried out in computer-mediated contexts should theoretically constitute CMDA. As Herring notes, a number of earlier CMC studies may be classified as CMDA, despite predating the formal existence of the approach (2004).

%Two major issues with Herring's definition are apparent. The first has previously been voiced by \textcite{jucker_linguistics_2012} who contend that `computer-mediated' is a misleading term, since many devices on which CMD can be produced (smartphones, tablets, etc.) are not commonly thought of as computers\endnote{Their proposed replacement term---\emph{keyboard to screen interaction}---suffers from even more obvious conceptual issues: many forms of CMD/CMD (e.g. \emph{Snapchat}, \emph{Skype}, etc.) are not reliant on keyboards.}.~Though this issue could be addressed with a replacement term such as `digitally-mediated discourse', CMC/CMD/CMDA are adopted as terminology in this thesis for the sake of convention\endnote{More problematic within Herring's definition, but less relevant to this thesis, is the stipulation that discourse is produced only between multiple human beings. This definition is far from robust, as it apparently excludes human-computer interaction (HCI). Already, humans communicate with bots in online video games, and bots may moderate content in chatrooms, discussion forums, and the like. Likewise, advertising discourse is a significant domain within discourse analytic research, and even within CMC \cite[e.g.][]{koteyko_multimodal_2007}. As human-computer interaction (HCI) becomes an increasingly important research area, and with much recent CA and SFL concerning HCI, it seems prudent to remove the human-human requirement of the definition.}.

%\subsection{Computer mediated discourse analysis}

%CMDA provides guidelines for operationalising and testing various theories upon data drawn from the Web. Guidelines for theoretical assumptions, research question formulation and classification of data are also given. Thus, Herring claims, CMDA research cannot invalidate CMDA itself, but simply point to areas in which more research is needed, and to areas that are difficult to implement. Instead, it is the operationalised theories of language and suitability of methodologies that are tested by CMDA research \cite{herring_computer-mediated_2001,herring_computer-mediated_2004,herring_computer-mediated_2011}.

%It is important to acknowledge that many studies within CMC qualify as CMD research, without having been expressly labelled as such. According to Herring, `any analysis of online behavior that is grounded in empirical, textual observations is computer-mediated discourse analysis' \textcite[p.~2]{herring_computer-mediated_2004}. That said, given the differences between CMC and CMD noted in the section above, `behavior' would be better replaced by `discourse': online behaviour terminologically connotes browsing history, search engine queries, etc., which cannot be said to constitute discourse in and of themselves.

%The earlier history of CMC provided at the beginning of this chapter is an important consideration when conceptualising CMDA, as the basic ideology of the approach is in a large part a reaction against the `first wave' of CMC research. The parameters of the approach have been designed to generate findings that demonstrate the heterogeneous nature of CMDs and elucidate the inherent power negotiations that shape them. At the same time, however, the CMDA approach is designed to address shortcomings of more contemporary research: Herring contends that the unsystematic, `ad-hoc' methodologies found in CMC research negatively affect the research area's overall explanatory power \citeyear{herring_computer-mediated_2004}.

%As an amalgamation of discourse analysis and CMC traditions, Herring explains that CMDA shares the two underlying assumptions of discourse analysis:

%\begin{enumerate}
%\item `that discourse exhibits recurrent patterns'
%\item `that discourse involves speaker choices'
%\end{enumerate}

%To these, Herring adds a third assumption, in order to draw attention to the computer-mediated nature of the data under investigation:

%\begin{enumerate}
%\setcounter{enumi}{2}
%\item `that computer-mediated discourse may be, but is not inevitably, shaped by the technological features of computer-mediated communication systems'
%\end{enumerate}

%The third assumption forms a contentious point in CMC theory: researchers such as \textcite{boyd_social_2007} contend that all CMD must be shaped by both the hardware and software on which the interaction was produced, while Walther's \citeyear{walther_computer-mediated_1996} arguments against a fundamental difference between FtF- and computer-mediated modes implicitly denies any deterministic influence of the technology itself. Here, an apparent contradiction emerges: as it is a major underlying tenets of modern discourse research that language and context are inseparable, it must be assumed that context on some level shapes the discourses produced within it. If this were not the case, elements of context could be removed without consequence on textual meaning---a position at odds with core values of functionalist approaches to language such as SFL, which contends that utterances simultaneously construct and respond to contextual conditions. Yet CMDA, despite subscribing this view, simultaneously rejects technological determinism, claiming that it is not inevitable that CMD will be shaped by situation factors. Indeed, the technologically deterministic approach is not well aligned with contemporary discourse research, especially when the focus is CMD, as it is reductive and deterministic, placing theoretical constraints on the range of possible utterances available in a given context.

%Taking a position of compromise, the orientation of this thesis is to reject outright technological determinism, but to concede technological \emph{shaping of} or \emph{influence over} the discourses found on the Bipolar Forum. As such, the overall message format and forum architecture are considered alongside users' posts as a part of the meaning-making process and a resource on which forum users may draw---though corpus linguistic research necessarily involves abstracting the language use in the forum and the way it is presented on-screen.