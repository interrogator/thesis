%!TEX root = ../thesis.tex
\section{Methodologies for investigating language in online support groups}

	In order to investigate online support group discourse socialisation, I draw upon three bodies of applied linguistic research. The first, \emph{computed mediated discourse analysi}s, is a framework for research design. In particular, the framework's \emph{faceted classification scheme} is operationalised for the purposes of site description. The second, \emph{corpus linguistics}, is a set of practices for creating and interrogating large digitised collections of text. The third, \emph{SFL}, is a theory of language that characterises language users as simultaneously drawing on three types of meaning---e\emph{xperiential}, \emph{interpersonal} and \emph{textual}---in order to achieve social goals. 

	In the sections below, I synthesise and critically reflect on relevant literature from these three domains, in order to provide a justification for the case study of this thesis, and its methodology, as described in Chapter 4.

	\subsection{Computer-mediated discourse analysis}

   		The first body of research is computer mediated discourse analysis (CMDA).  CMDA is used to provide a framework for site conceptualisation and description, as well key theoretical assumptions concerning the relationship between discourse and computer mediation. In the sections below, I contextualise CMDA within computed mediated communication research and outline relevant tenets of the framework as provided by Herring \citeyear{herring_computer-mediated_2001,herring_computer-mediated_2004,herring_faceted_2007,herring_discourse_2011,herring_computer-mediated_2011}.

   		%~\\\todo[inline,color=green!40]{More needs to be done to integrate the discussion of CMC/CMD/CMDA with online socialisation literature from the previous chapter.}

		\subsubsection{Computer-mediated communication}

			%CMDA is best understood within the context of CMC and CMC research.

			CMC research has a rich and multifaceted history that colours much contemporary academic research into discourse produced online. As reasons of space preclude full treatment of the evolution of CMC and CMC research (see REFERENCE for an in-depth account), I summarise here only the developments that may be seen as necessitating the development of CMDA.

			%The earliest research into CMC typically treated CMC from a defecit model: as mediated communicators were unable draw upon non-verbal communication, essentially confined to written language, online interactions were seen as inherently impoverished, when compared to face-to-face modes. The proposed consequences of the `reduced cues' were at times contradictory. One one hand, CMC was argued to be too impersonal to facilitate the development of close relationships, and the anonymity afforded to interactants was seen as allowing them to communicate with fewer inhibitions, causing \emph{flaming}. On the other hand, the \emph{idealised other} theory suggested that the lack of signifiers of difference would cause Internet users to treat online strangers with the assumption of sameness, and thus kindness. Related was the idea that people online were able to locate communicative partners based on shared interests and values, unbound by geographical location, with the potential consequence of more meaningful exchanges.

			%From the 1970s to the early 1990s, CMC was largely limited to specialised communities, with technologies such as ARPANET designed with military and research purposes foremost in mind \cite{thorne_computer-mediated_2008}. As the Internet user-base gradually diversified to include university staff and students, as well as select members of the private sector, CMC became progressively more social in nature. By the early 1980s, linguists were researching both synchronous (i.e. chat) and asynchronous (e.g. forums, email) digital environments \cite[e.g.][]{carey_paralanguage_1980,myers_anonymity_1987,pullinger_chit-chat_1986}.
	  
			%This early body of research posited enduring representations of computer-mediated and online environments \cite{postmes_formation_2000}. Often articulated was the \emph{reduced-cues perspective} \cite{thorne_computer-mediated_2008}, which argued that CMC was a ``lean'' or ``low-bandwidth mode'', stripped of the kinds of context available in face-to-face (FtF) settings. Due to the observation that ``social cues are filtered out in on-line settings'' \cite[p.~81]{parks_making_1996}, for instance, it was argued that complex corporate information was better suited to FtF transmission, due to the comparative ``richness'' of the FtF mode. Computer-mediated channels, under this model, were seen as best reserved for dealing with large sets of quantitative data \cite{daft_information_1983}. Similarly, in a social scenario, Parks and Floyd contended that the unavailability of ``information regarding physical appearance'' (1996, p. 84) stymied the potential for users to build meaningful social relationships online.

			Given the novel mixture of intimate and detachment occurring during text-based CMC \cite{king_researching_1996}, research into the affordance of anonymity and its ramifications also proved major themes of early literature \cite{tanis_two_2007}. It was commonly argued until as late as the mid 1990s that anonymous CMC was inherently genderless and egalitarian, due to the difficulty of verifying the identity and credentials of participants \cite{herring_computer-mediated_2001}. Furthermore, researchers consistently argued that participants' anonymity allowed for reduced inhibitions, causing an increase in hostile discourses, or \emph{flaming} (Collins, 1992). \textcite{kiesler_social_1984} contended that flaming arose due to a lack of audience feedback, an inability to control discourse and ``depersonalization'' stemming from an absence of non-verbal communication. \textcite[p.~7]{kim_verbal_1991} concurred, stating that in CMC, participants were more likely ``to abuse, make offensive comments, or criticize sharply''.

			\paragraph{The contemporary digital landscape}

				Between the mid 1990s and the mid 2000s, the nature of CMC---and consequently, CMC research---underwent a period of rapid change. As the result of a number of interwoven factors (the growing affordability of home-computers; increasingly digital literacy; the development of early social network sites, etc.) the landscape of the Internet shifted from being primarily static and text-based to dynamic, multimodal and participatory in nature \cite{herring_discourse_2011,lindholm_identity_2012}. The three currently most popular websites according to \emph{Alexa} (\emph{Google}, \emph{Facebook} and \emph{YouTube}) exemplify this shift, in that each allows a great deal of user-input and provides content in textual, audio and graphic modes.

				The profound nature of this change has meant that much early CMC literature has now lost some of its relevance. Research into email messages is problematised by the fact that social media has overtaken email as the primary CMC interface for most digitally literate citizens \cite{thorne_computer-mediated_2008}. Similarly, the anonymity posited as critical to the discourse of early CMC, while still possible, is no longer the norm: on social networking sites (SNSs) such as \emph{Facebook}, users communicate through personal accounts, disclosing their identities as they communicate and often even rendering the communication viewable to family and friends \cite{boyd_social_2007}. Furthermore, given that SNSs are multimodal, and often contain a mixture of synchronous and asynchronous discourses, research informed by a characterisation of the Internet as fundamentally text-based, or dealing exclusively with synchronous or asynchronous datasets, is of limited usefulness for researchers of multimodal CMC\endnote{For example, the myriad studies of communication on \emph{Usenet}---the text-only Internet communication system in which many of the social elements of CMC were first popularised---have been made largely redundant by the decommissioning of Usenet servers in 2010 \cite[e.g.][]{berge_computer-mediated_1995,eklundh_use_1994,jaffe_gender_1995} That said, researchers have noted that despite greater bandwidth and the potential for multimodal interaction, in many respects, CMC has remained surprisingly text-based.}. 

				In addition to the problem of reduced applicability, the findings of earlier research have also been challenged by those of more recent investigations \cite{herring_computer-mediated_2001,postmes_formation_2000}. The characterisation of CMC as impersonal, ineffective and hostile nature has been questioned by \textcite{walther_computer-mediated_1996}, who argues that many early findings were caused by researchers’ placing time restrictions on the observed CMC interactions. If CMC interactions are given unlimited time, Walther contends, they achieve the same level of depth as FtF scenarios, both in terms of task-completion and in the development of social relationships. From this perspective, aside from the slower pace of interactions, synchronous text-based CMC and face-to-face communication can be conceptualised as fundamentally similar in nature \cite{walther_interpersonal_1994,osullivan_reconceptualizing_2003,wu_is_2013}.

				The notion of a genderless and egalitarian Internet has faced similar scrutiny: CMC research has shown that gender in anonymous, text-only CMC is often encoded by the \glslink{lexicogrammar}{lexicogrammatical} choices of the writer, as well as through the use of gendered discourse strategies such as assertiveness, politeness and aggression \cite{herring_gender_2000}. Likewise, education level is conveyed through vocabulary and complexity of message structure, and age through the discussion of interests and life experiences \cite{herring_computer-mediated_2001}. Finally, research from the perspective of critical discourse analysis has questioned the notion of an egalitarian Internet at two separate levels. At the level of discourse, newer studies show that complex and rigid power structures exist online, even in low-bandwidth forums and discussion lists \cite[e.g.][]{stommel_online_2010}. More broadly, contemporary theorists acknowledge that the landscape of CMC has ``inherit[ed] power asymmetries from the larger historical and economic context of the Internet'', with a notable over-representation of English speaking, white males positioned as moderators, webmasters, and page creators \cite[p.~12]{herring_computer-mediated_2001}.

			\paragraph{Computer mediated discourse}

				Herring has defined computer-mediated discourse as ``the communication produced when human beings interact with one another by transmitting messages via networked computers'' \cite[n.p]{herring_computer-mediated_2001}. Thus, CMD very often overlaps with CMC. There are some key differences, however. First is that a single word or sentence sent through an online channel constitutes CMC. As discourse extends beyond the sentence barrier and potentially between multiple participants, a sample must be larger in order to qualify as CMD. Similarly, as discourse constructs and is reflexively constructed by its context, CMD data must be language in context, whereas CMC may still be considered as such when it has been decontextualised (by copy and pasting, for example). Finally, as context in online communication is almost always rendered multimodally (avatars and hyperlinks; font and background colours, etc.), it logically follows that CMD, as contextualised language, must therefore inherently be concerned with multimodality to some extent, while CMC data may consist purely of written text.

				%We can surmise from this that DA and CMDA are congruent: DA carried out in computer-mediated contexts should theoretically constitute CMDA. As Herring notes, a number of earlier CMC studies may be classified as CMDA, despite predating the formal existence of the approach (2004).

				Two major issues with Herring's definition are readily apparent. The first has previously been voiced by \textcite{jucker_linguistics_2012} who contend that ``computer-mediated'' is a misleading term, since many devices on which CMD can be produced (smartphones, tablets, etc.) are not commonly thought of as computers\endnote{Their proposed replacement term---\emph{keyboard to screen interaction}---suffers from even more obvious conceptual issues: many forms of CMD/CMD (e.g. \emph{Snapchat}, \emph{Skype}, etc.) are not reliant on keyboards.}.~Though this issue could be addressed with a replacement term such as ``digitally-mediated discourse'', CMC/CMD/CMDA are adopted as terminology in this thesis for the sake of convention\endnote{More problematic within Herring's definition, but less relevant to this thesis, is the stipulation that discourse is produced only between multiple human beings. This definition is far from robust, as it apparently excludes human-computer interaction (HCI). Already, humans communicate with bots in online video games, and bots may moderate content in chatrooms, discussion forums, and the like. Likewise, advertising discourse is a significant domain within discourse analytic research, and even within CMC \cite[e.g.][]{koteyko_multimodal_2007}. As human-computer interaction (HCI) becomes an increasingly important research area, and with much recent CA and SFL concerning HCI, it seems prudent to remove the human-human requirement of the definition.}.

		\subsubsection{Computer mediated discourse analysis}

			CMDA provides guidelines for operationalising and testing various theories upon data drawn from the Web. Guidelines for theoretical assumptions, research question formulation and classification of data are also given. Thus, Herring claims, CMDA research cannot invalidate CMDA itself, but simply point to areas in which more research is needed, and to areas that are difficult to implement. Instead, it is the operationalised theories of language and suitability of methodologies that are tested by CMDA research \cite{herring_computer-mediated_2001,herring_computer-mediated_2004,herring_computer-mediated_2011}.

			It is important to acknowledge that many studies within CMC qualify as CMD research, without having been expressly labelled as such. According to Herring, ``any analysis of online behavior that is grounded in empirical, textual observations is computer-mediated discourse analysis'' \textcite[p.~2]{herring_computer-mediated_2004}. That said, given the differences between CMC and CMD noted in the section above, ``behavior'' would be better replaced by ``discourse'': online behaviour terminologically connotes browsing history, search engine queries, etc., which cannot be said to constitute discourse in and of themselves.

			The earlier history of CMC provided at the beginning of this chapter is an important consideration when conceptualising CMDA, as the basic ideology of the approach is in a large part a reaction against the ``first wave'' of CMC research. The parameters of the approach have been designed to generate findings that demonstrate the heterogeneous nature of CMDs and elucidate the inherent power negotiations that shape them. At the same time, however, the CMDA approach is designed to address shortcomings of more contemporary research: Herring contends that the unsystematic, ``ad-hoc'' methodologies found in CMC research negatively affect the research area's overall explanatory power \citeyear{herring_computer-mediated_2004}.

			As an amalgamation of discourse analysis and CMC traditions, Herring explains that CMDA shares the two underlying assumptions of discourse analysis:

			\begin{enumerate}
			\item ``that discourse exhibits recurrent patterns''
			\item ``that discourse involves speaker choices''
			\end{enumerate}

			~\\\todo[inline,color=green!40]{Missing is an explain the ramifications of these. Perhaps find a quote from Firth or Fairclough for the first assumption; for the second, Herring cites Sacks (1984) as saying that cognitive and social factors may affect discourse as do speaker choices}

			%COMMENT These parameters seem designed with research implications in mind: the former lends generalisability to properly-constructed investigations; the latter 

			To these, Herring adds a third assumption, in order to draw attention to the computer-mediated nature of the data under investigation:

			\begin{enumerate}
			\setcounter{enumi}{2}
			\item ``that computer-mediated discourse may be, but is not inevitably, shaped by the technological features of computer-mediated communication systems''
			\end{enumerate}

			The third assumption forms a contentious point in CMC theory: researchers such as \textcite{boyd_social_2007} contend that all CMD must be shaped by both the hardware and software on which the interaction was produced, while Walther's \citeyear{walther_computer-mediated_1996} arguments against a fundamental difference between FtF- and computer-mediated modes implicitly denies any deterministic influence of the technology itself. Here, an apparent contradiction emerges: as it is a major underlying tenets of modern discourse research that language and context are inseparable, it must be assumed that context on some level shapes the discourses produced within it. If this were not the case, elements of context could be removed without consequence on textual meaning---a position at odds with core values of functionalist approaches to language such as SFL, which contends that utterances simultaneously construct and respond to contextual conditions. Yet CMDA, despite subscribing this view, simultaneously rejects technological determinism, claiming that it is not inevitable that CMD will be shaped by situation factors. Indeed, the technologically deterministic approach is not well aligned with contemporary discourse research, especially when the focus is CMD, as it is reductive and deterministic, placing theoretical constraints on the range of possible utterances available in a given context.

			Taking a position of compromise, the orientation of this thesis is to reject outright technological determinism, but to concede technological \emph{shaping of} or \emph{influence over} the discourses found on the Bipolar Discussion Forum. As such, the overall message format and forum architecture (see Section \ref{sect:medfact}) are considered alongside users' posts as a part of the meaning-making process and a resource on which forum users may draw.

		\subsubsection{Research design}

			  ~\\\todo[inline,color=green!40]{Research question formulation is provided in order to clarify what phenomena can be suitably explored with CMDA is provided}

%\paragraph{Data collection}

			%The CMDA framework also provides basic instructions for data sampling. Most importantly, Herring explains, data must of appropriate in size and content to answer research question(s), and as rich in context as possible. Noting the wealth of content often afforded to researchers of CMD, Herring argues that manageable datasets are best obtained through sampling.
		 
			%\begin{table}
			%\centering
			%\includegraphics[scale=.5]{../../images/sampling.png}
			%\caption[Data sampling techniques for CMDA]{Data sampling techniques for CMDA (From Herring, 2004)}
			%\label{tab:sampling}
			%\end{table}

			%Temporal or thematic sampling (see Table \ref{tab:sampling}) are considered ideal \cite{herring_computer-mediated_2004}, as they provide the most coherent discourse. This is in preference to random sampling, which inherently decontextualises selected samples from those surrounding it. That said, temporal and thematic sampling may also truncate text or remove potentially important contextual elements, as references may be made within texts to conversations that occurred outside of the chosen sampling period. 

			%Within standard web forum architecture, threads are organised by the timestamp of the most recent post. While active, threads may thus shift in relation to one another, becoming more or less fixed in position only when no longer active. The ramification of this is that the context of adjacent threads may differ substantially for forum users and researchers: for the former, thread position is unstable and reactive to their posts; for the latter, who typically investigate threads after their development has ceased, threads falsely appear stable, and reconstruction of thread movement is often not possible.

			%The data sampling approach undertaken in this thesis is variously by theme, phenomenon and individual, depending on the focus of the investigation. The first investigation involves the collection of the entire forum, with analysis involving the grouping of keywords into discourses. The second investigation is by \emph{theme}, in that all first posts to the forum are sampled. In the third case, post histories of selected \emph{individuals} are harvested to track the process of socialisation. In each case, however, in contrast to Herring's suggested methods, sampling is data-driven (see Section \ref{sect:shortcl}), in that preliminary analysis of the forum from a quantitative perspective informs the later sampling decisions. The benefits and consequences of the approach are weighed in Chapter 4.

		 \paragraph{Analytical methods}

			%\begin{table}
			%\centering
			%\includegraphics[scale=.5]{../../images/fourdomains.png}
			%\caption[Domains of language and relevant analytical methods]{Domains of language and relevant analytical methods (from Herring, 2004)}
			%\label{tab:domains}
			%\end{table}

			%~\\\todo[inline,color=green!40]{Explanation of domains of language here.}

			%For the purposes of determining relevant linguistic features for analysis, Herring divides language into four ``domains'', each of which is best researched through the invocation of selected analytical paradigms, as shown in Table \ref{tab:domains}. Added to these is the domain of ``participation'', which is realised in the online environment as the number of messages, thread lengths, etc., which may be of interest to a linguistic analysis but according to Herring do not constitute linguistic phenomena in and of themselves\endnote{This point would perhaps be contested by \textcite{eggins_analysing_2004}, whose approach is more or less adopted in this thesis. Shortcomings of Herring's division of language into domains are presented later in Section \ref{sect:shortsfl}.}.~Examples of linguistic realisations of the domains are also provided, but are not to be treated as exhaustive: more may be added upon being discovered empirically within CMD.

			%As the time it would take to code relevant phenomena from all domains is often not feasible, producing more results than may be presented in an average journal article, Herring advocates ``select[ing] those features to code that she believes will produce the most valid and convincing results in relation to the research question'' \cite[p.~20]{herring_computer-mediated_2004}.

		\subsubsection{Faceted classification}

			As an addition to the CMDA framework, \textcite{herring_faceted_2007} provides a preliminary scheme for the classification of CMC environments. Based on empirical evidence from existing CMC/CMDA literature, she contends that CMD is constrained by \emph{medium} and \emph{situation} factors. Medium factors are technological; situation factors are social. These factors are then subdivided into non-restrictive categories, each of which an analyst may discuss based on their perceived influence over the communication taking place within the environment. According to Herring, the goal of the scheme is ``to synthesize and articulate aspects of technical and social context that influence discourse usage in CMC environments'' (n.p). The scheme is thus useful for both enhancing researchers' conceptualisations of their own data and for the purpose of rendering other CMDA findings cross-applicable by standardising terminology and description of relevant areas of CMD sites.

		 	Facets have no inherent hierarchical relationship with one another, but are not necessarily equally influential. Herring explains that ``the relative strength of social and technical influences must be discovered for different contexts of CMD through empirical analysis''. % LINK TO RQ? - ROBYN

		 	\paragraph{Medium factors}\label{sect:medfact}
		 
				Herring identifies 10 empirically observed factors that may influence CMD at the technological level. They are reproduced in Table \ref{tab:mediumfactors}, alongside a definition in question form and an answer for the case of the Bipolar Forum.

				\begin{table}\small{
				\begin{tabularx}{420pt}{|L{0.15}|L{0.65}|L{1.4}|L{1.8}|}
				\hline
				\textbf{\#} & \textbf{Medium factor}			 & \textbf{Definition}															   & \textbf{Healthboards Bipolar Forum}																																									\\ \hline
				1  & \sctext{Synchron-icity}			 & Must users be simultaneously logged in to receive messages?			  & No. More and less synchronous interactions may occur																															\\ \hline
				2  & \sctext{Message transmission}	  & Message-by-message or character by character?							& Message by message																																							  \\ \hline
				3  & \sctext{Persistence of transcript} & For how long does the message remain accessible?						 & Indefinitely. That said, topics are `closed' after a certain period of time with no activity.																				   \\ \hline
				4  & \sctext{Size of message buffer}	& What is the longest possible single message?							 & Essentially limitless: larger than would be comfortable to type or read.																										\\ \hline
				5  & \sctext{Channels of communication} & Text, audio, video, images?											  & Very much text-based, with emoticons and off-site links forming the most obvious multimedia.																					\\ \hline
				6  & \sctext{Anonymous messaging}	   & What level of anonymity does the site provide at a technological level?  & At technological level, usernames are used, and email addresses may be recoverable.																							 \\ \hline
				7  & \sctext{Private messaging}		 & Do users have the ability to write private messages to one another?																		&  Yes. These messages are not available for analysis, and potentially, members socialise and are socialised through private messages as well. \\ \hline
				8  & \sctext{Filtering}				 & Can users ignore messages of others?									 & Private messages, yes. Forum posts, no.																																		 \\ \hline
				9  & \sctext{Quoting}				   & Is there an inbuilt mechanism for quoting other users' messages?		 & Yes. This allows users to echo sentiments or enhance readability when their response is to an earlier post in the thread														\\ \hline
				10 & \sctext{Message format}			& Where do new messages appear, and what happens when messages are filled? & New thread appears on top of the `board'. Newest posts at the bottom of threads. New page after n replies																	   \\ \hline
				\end{tabularx}
				}
				\caption[Medium factors in the faceted classification scheme]{Medium factors (adapted from  Herring, 2007)}
				\label{tab:mediumfactors}
				\end{table}

				The most relevant of these for the bipolar forum investigations in this thesis is \sctext{Message format}, which essentially refers to web forum architecture in general (that is, the organisation of threads and posts with distinct titles, timelines, beginnings and ends). Message format options on the forum are hypothesised in the investigations of this thesis to be an important meaning-making resource for forum members.

				Of secondary interest is the relative \sctext{anonymity} afforded to posters on the board, as this has been observed to lead to increased disclosure in OSGs. The discussion of Investigation C, which tracks members longitudinally, reflects on decreased anonymity over time as a factor influencing shifts toward the normative values of the forum.

			\paragraph{Situation factors}

				Situation factors are socially constituted: other scenarios would be possible with the same technological factors. Table \ref{tab:situationfactors} summarises Herring's proposed facets and gives examples.

				\begin{table}[ht] \singlespacing \small{
				\begin{tabularx}{\textwidth}{|L{0.05}|L{0.20}|X|}
				\hline
				\textbf{\#} & \textbf{Situation factor}			 & \textbf{Examples}						  \\ \hline
				1  & \sctext{Partici-pation structure}		  & \compress \begin{itemize} \item One to one, one-to-many, many-to-many \item Public/private \item Degree of anonymity \item Group size, number of active participants \item Amount, rate and balance of participation \end{itemize}		   \\ \hline
				2  & \sctext{Participant character-istics}	  & \compress \begin{itemize} \item Demographics \item Language/computer proficiency \item Experience with topic  \item Role/status on/offline \item Pre-existing knowledge and interactional norms \item Attitudes, beliefs, ideologies, and motivations \end{itemize}		 \\ \hline
				3  & \sctext{Purpose}						  & \compress \begin{itemize} \item Of group, e.g. professional, social, fantasy/role-playing \item Goal of interaction, e.g. get information, negotiate consensus \end{itemize}	  \\ \hline
				4  & \sctext{Topic/ theme}					  & \compress \begin{itemize} \item Of group, e.g. politics, linguistics, feminism ... \item Of exchanges, e.g. the war in Iraq, pro-drop languages, etc. \end{itemize}	\\ \hline
				5  & \sctext{Tone}							 & \compress \begin{itemize} \item Serious/playful \item Formal/casual \item Contentious/friendly  \item Cooperative/sarcastic \end{itemize}					   \\ \hline
				6  & \sctext{Activity}						 & \compress \begin{itemize} \item E.g. debate, job announcement, information exchange, phatic exchange, problem solving, exchange of insults, joking exchange, game, theatrical performance, flirtation, virtual sex \end{itemize}	\\ \hline
				7  & \sctext{Norms}							& \compress \begin{itemize} \item Of organisation \item Of social appropriateness \item Of language \end{itemize}			   \\ \hline
				8  & \sctext{Code}							 & \compress \begin{itemize} \item Language, language variety \item Font/writing system \end{itemize}							 \\ \hline
				\end{tabularx}}
				\caption[Situation factors in the faceted classification scheme]{Situation factors (adapted  from Herring, 2007)}
				\label{tab:situationfactors}
				\end{table}

				Given that the site of the investigation is a relatively prototypical message board, it is the situation factors that more usefully differentiate it from investigations of similar modes. Below are brief descriptions of each situation factor.
			
			\begin{enumerate}
			\small{
			\singlespacing{
			\item \sctext{Participant structure:} One-to-many for first posts, followed generally by many-to-one replies. Most interactions are public, but relatively anonymous: only occasionally do users provide a given name, and family names are almost absent. Private messaging is possible (but not accessible as data). Generally, a username, avatar, and occasionally gender and location information is provided. Group size changes over time, and it is difficult to determine who is active, since most posters contribute a total of fewer than five posts. Average number of replies to posts: n
			\item \sctext{Participant characteristics:} predominantly North American user-base, with other English speaking countries well-represented, and small minorities of ESL users. Female to male ratio is approx 4:1. Language and computer proficiency vary: some posts demonstrate poor spelling and grammar and unfamiliarity with ``netspeak'' conventions. Users do not know one another outside of the forum. Most either have BPD or are a family member/friend/partner/ex-partner of someone with BPD
			\item \sctext{Purpose:} Health information and social support exchange; empowerment
			\item \sctext{Topic:} Bipolar disorder and subtypes. Individual threads cluster around diagnosis, symptoms, medication, personal narratives, issues with doctors, etc.
			\item \sctext{Tone:} Generally serious, supportive, semi-formal, with joking, sarcasm evident among veteran members and lighthearted threads
			\item \sctext{Activity:} various, but broadly, information and support seeking
			\item \sctext{Norms:} New users often introduce themselves in a new thread; veteran users welcome new users; moderators' authority respected; little disagreement
			\item \sctext{Code:} Only English, plus ``netspeak'', medical terminology, and slang medical terminology (e.g. \emph{meds})
			}
			}
			\end{enumerate}
			
	  \subsubsection{Key issues}

		 A key strength of CMDA is its robustness: in not being reliant on any single theory of language, the tenets of the approach may be disproved through empirical research. Instead, research simply demonstrates either the usefulness of the established principles of the approach, or the need for further development or refinement of its tenets \cite{herring_computer-mediated_2001}.

		 In terms of refinement, verbose and unwieldy elements of the faceted classification are an acknowledged problem \cite{herring_faceted_2007}. An informal search of CMDA papers found that few have operationalised the faceted classification scheme to any significant extent \cite[e.g.][]{chiluwa_citizenship_2012,lindholm_identity_2012}, and no study was found that explicitly takes up Herring's call for empirical analysis of relationships between facets. As Herring advocates the use of only relevant elements of the scheme, this issue is not a particularly alarming one. Nonetheless, it points to the need to collapse categories where possible and according to empirical site classification.

		  \paragraph{Methodology: lack of systematic, quantitative methods}\label{sect:shortcl}

			A major shortcoming within CMDA concerns the lack of  web-corpus linguistic methods. Most CMDA research to date has involved the collection of single or very small sets of interactions, as is the norm in FtF CA research \cite{androutsopoulos_introduction:_2008}. Similarly, for studies of virtual community, it has been more common thus far to draw on anthropological techniques of observation (in CMC environments, lurking) combined with recording (copy and pasting of text-based communication, or screengrabbing of real-time multimodal interaction), rather than collecting all possible content and interrogating it as a whole for quantitative results. This is unfortunate, given the natural suitability of many kinds of CMD for quantitative linguistic research: much is indexed, searchable, constantly updating, diverse, easily stored, and the amount of accessible data is potentially limitless \cite{baroni_wacky_2009}.

			Exploitation of these common features of CMC would foreseeably improve CMDA's current guidelines for data sampling. Currently, each noted sampling technique takes place before any data analysis occurs (see Table \ref{tab:sampling}). In this way, a key resource is discarded without ever being drawn upon. Where possible (i.e. when sufficient data exists and can feasibly be collected), it seems sensible to instead involve the aggregated data in the sample selection process, as is often done in CADS: if researchers attempt to first harvest \emph{all} relevant language within the environment, rather than sampling, basic corpus linguistic interrogation could reveal insights that may inform the selection of texts in a systematic, data-driven manner. Keywords and key clusters can be concordanced; concordances exemplifying the quantitatively significant phenomenon can be linked to the contextualised data, which can undergo qualitative analysis. This enhances researchers' ability to claim representativeness of their sample and generalisability of their findings.

			As the transformation of large quantities of natural language online into corpus data is easily achievable with techniques considered rudimentary within corpus building, NLP and computer science, this thesis introduces tenets of these approaches---in particular, corpus linguistics---to CMDA. Chapter 4 discusses basic corpus linguistic theory and outlines the successes of corpus linguistic approaches to discourse research within the emerging field of CADS. Chapter 10 establishes guidelines for larger-scale data collection.

		 \paragraph{Lack of a theory of language and context}\label{sect:shortsfl}

			It is at this point perhaps worth questioning the value of the purported theory-neutrality of the CMDA approach. In not relying on the knowledge generated within well-established linguistic paradigms, CMDA must instead to some degree speculate as to the ``domains of language''. The subsequent division of language into \emph{structure, meaning, interaction} and \emph{social behaviour} may be accused of not only being inherently theoretical, but of also oversimplifying the interrelationship between linguistic systems, especially when the division does not appear to be empirically grounded.

			%ROBYN: More specific element of SFG, e.g. Eggins and Slade

			The ``domains of language'' approach points to what is perhaps a misunderstanding concerning the nature of a functionalist approaches to language: while grammars such as that provided by systemic functional linguistics stress the interdependent nature of linguistic systems and provide a grammar that highlights these dependencies, CMDA implicitly segregates them by advocating the use of different analytical paradigms for each ``domain''. Herring's suggestion that CDA practitioners, for example, are concerned with unspecific linguistic phenomena such as ``expressions of status, conflict, negotiation, face-management [or] play'' \citeyear[p.~18]{herring_computer-mediated_2004} is at odds with the wealth of CDA research investigating the ways in which linguistic phenomena such as turn-taking, semantic associations or syntax are used to reproduce societal power structures and inequalities \cite{wodak_methods_2009}. CMDA's existing compartmentalisation of micro-, meso- and macro-level linguistic phenomena could thus foreseeably obscure the ways in which meaning is simultaneously conveyed at different linguistic strata.

			To overcome this issue, I propose SFL as a candidate for addition to the analytical methods of the CMDA approach. In many respects, CMDA and SFL are an intuitively good fit, as they share a number of underlying similarities. Both are concerned with empirical research of language in context, and both seek to make discourse opaque through linguistic analysis. Moreover, both view language at its core as functional, social and semiotic. Indeed, the addition of new theories of languages to CMDA is encouraged by Herring, who explains that the analytical paradigms listed in \citeyear{herring_computer-mediated_2004} are simply those she is most familiar with.

			Chiefly, SFL provides justification for the division of language into domains, while also providing an account of the ways in which these domains may relate to one another.  As will be discussed in the next chapter, the three metafunctions within a systemic functional grammar pattern to a reasonable extent with Herring's domains of language: \emph{structure}, \emph{meaning} and \emph{interaction} respectively echo the \emph{textual}, \emph{experiential} and \emph{interpersonal} dimensions of language outlined by \textcite{halliday_language_1994}. That the SFL categories have been well-researched and empirically observed is of great potential benefit to CMDA, and SFL's provision of a coherent grammar ``help[s] reduce the arbitrariness of interpretation [within CMDA] by anchoring it in the linguistic form itself'' \cite[p.~23]{galasinski_fathers_2013}. This is in line with Herring's statement that an overall goal of CMDA research is ``to enable questions of broad social and psychological significance, including notions that would otherwise be intractable to empirical analysis, to be investigated with fine-grained empirical rigor'' \cite[p.~2]{herring_computer-mediated_2004}.

			Though an increasing amount of SFL research has engaged with CMC, a search of the literature revealed few studies drawing explicitly on CMDA. 

			~\\\todo[inline,color=green!40]{Here I will add a basic review of \\ Abbasi, A., \& Chen, H. (2008). CyberGate: a design framework and system for text analysis of computer-mediated communication. \\ Coffin, C. (2013). Using systemic functional linguistics to explore digital technologies in educational contexts.  \\ Zhuanglin, H. (2008). How is Meaning Construed Multimodally? A Case Study of a PowerPoint Presentation.}

			The few papers utilising perspectives from both show promise, but ultimately point to the need for framework development, rather than ad-hoc implementation of individual components of either research area. This thesis provides basic parameters for the integration of the approaches in Chapter 9.

	\subsection{Systemic functional linguistics}

		%intro
		%outline and suitability
		%constitutive of context
		%meaning-making resource
		%interpersonal and ideational?
		%interpersonal
			%clause type
			%modalisation and modulation
			%appraisal
		%transitivity
			%process types
			%kinds of participants

		%theme?
		%genre
		%
		%

	  % lexis and grammar are not separate modules or components, but merely zones within a continuum \cite{halliday_introduction_2004}

	  Given the fact that CMDA's proposed analytical methods are not exclusive, and that Herring welcomes the use of other theories of language within the approach, the analysis presented in this thesis relies upon systemic functional linguistics---specifically, the experiential and interpersonal metafunctions (including \emph{appraisal}), and the notion of genre as delineated by \textcite{hovy_types_1996,martin_analysing_1997,martin_genre_2006} and as operationalised by \textcite{eggins_analysing_2004}. 

	  In the sections below, I provide an outline of SFL as a means of investigating language in use. Following this, I provide a description of the interpersonal and experiential metafunctions, as well as systemic functional treatment of genre.

	  \subsubsection{Outline and suitability of the theory}

	  Developed chiefly by Halliday \citeyear{halliday_language_1994,halliday_introduction_2004}, SFL is a functional theory of language and context influenced by Saussurean semiotics, Manilowski, and the work of J.R. Firth \cite{halliday_language_1989}. It takes from de Saussure the notion of linguistic units as signifiers, but also expands upon the Saussurean position by considering language not as a study of signs, but of \emph{sign-systems}---``not sets of individual things, but rather networks of relationships'' \cite[p.~4]{halliday_language_1989}. From Malinowski, SFL derives its division of context into a \emph{cultural dimension}, which shapes all interactions taking place within the culture, and a \emph{situational dimension}, which concerns the specific environment in which a given text is produced \cite[p.~6]{halliday_language_1989}. Finally, Firth's elaboration and linguistic re-orientation of the context of situation as fundamentally concerning participants, actions and circumstances informs SFL's division of lexicogrammar into three metafunctions \cite[p.~8]{halliday_language_1989}.

	  %That is, SFL does not offer a unitary, coherent theory but rather a collection of studies by scholars who originally have been inspired by the work of Halliday, and who still use some of the standard notions of SFL, but who otherwise have gone their own way, as is for example the case of such varied approaches as those of Jim Martin, Eija Ventola, Jay Lemke, Norman Fairclough and Theo van Leeuwen, among many others. \cite{van_dijk_text_2004}

	  %`meaning, in its most general sense' \cite[p.~4]{halliday_language_1989}.

	  As a functional theory of language, SFL may be characterised as being in opposition to generativist or formalist linguistic traditions, which focus on \emph{competence}, and use hypothetical and ``possible'' utterances to determine grammatical rules. Functionalist strains of linguistics instead rely on \emph{performance}, studying instances of real language as data. SFL is concerned with exploring `how people use language' in the real world \cite[p.~2]{eggins_introduction_2004}. A further characteristic of functionalist approaches is that they foreground the notion that the functions of language motivate its grammar, in much the same way that the shape of a physical tool is influenced by its uses \cite{nichols_functional_1984}. Accordingly, a central concern of SFL research is in determining `how language is structured for use' \cite[p.~2]{eggins_introduction_2004}. 

	  SFL is differentiated from other branches of functional linguistics by its commitment to the idea that all strata of language (phonetics, lexis, semantics, etc.) exist as parts of an overall linguistic system that is drawn upon by language users to achieve social goals \cite{halliday_language_1994}. The provision of a detailed grammar capturing the interrelationship of linguistic sub-systems, while necessary for such an undertaking, is also unique to the approach \cite{eggins_introduction_2004}. Central is the concept of rank scale, which corresponds with strata of language. A single instance of each rank is comprised by one or more instances of the rank(s) below.

	  \begin{table}[ht]
	  \centering
		  \begin{tabular}{|l|}
		  \hline
		  Clause complex \\ \hline
		  Clause		 \\ \hline
		  Group/phrase   \\ \hline
		  Word		   \\ \hline
		  Morpheme	   \\ \hline
		  \end{tabular}
		  \caption{Rank Scale}
	  \end{table}

	  %Moving down the scale from system, the meaning potential of the language as a whole becomes progressively narrowed, beginning with genres and registers, and narrowing further with text types, until we arrive at an individual text (an instance of language) \cite{martin_genre_2006}

	  In SFL, language users are seen as making choices at the level macro strata that place constraints on what can be realised at lower strata \cite{martin_genre_2006}. After a choice of semantic meaning is made, for example, the speaker is presented with congruent and incongruent realisations of the semantic meaning within the lexicogrammar. Incongruent realisation, or \emph{grammatical metaphor}, is the mechanism by which language can have limitless meaning potential \cite{heyvaert_nominalization_2003}: through this device, `atypical' realisations of one rank through another (for example) may be chosen by speakers to satisfy experiential, interpersonal or textual goals. 

	  %Experientially, for example, grammatical metaphor may allow a MESS; interpersonally, modality allows hedging \cite{heyvaert_nominalization_2003}.

	  %An element of linguistic competence is the ability to recognise the unmarked form \cite{heyvaert_nominalization_2003}.
	  %Goal oriented: this assumption explains the provision of unsolicited bipolar advice!

	  SFL has been applied to a diverse array of research areas including language pedagogy \cite{halliday_towards_1993}, historical linguistics \cite{martin_re/reading_2003}, language acquisition \cite{hasan_learning_1994} and (critical) discourse analysis \cite{hunston_systemic_2013,le_systematic_2009}. The latter of these---the research area of this thesis---is particularly well-suited to SFL: as Halliday explains, at the most general level, SFL is used ``to understand the quality of texts: why a text means what it does, and why it is valued as it is'' \citeyear[p.~xxx]{halliday_introduction:_2004}. The provision of a grammar for discourse analysis is a key contribution of SFL within applied linguistics. In fact, from an SFL perspective, the use of a grammar is a prerequisite for all discourse analytic research. In Halliday's words,

	  \begin{quote}\small\singlespacing
	  it is sometimes assumed that discourse analysis, or `text linguistics' can be carried on without grammar---or even that it is somehow an alternative to grammar. But this is an illusion. A discourse analysis that is not based on grammar is not an analysis at all, but simply a running commentary on a text \citeyear[p.~xvii]{halliday_introduction:_2004}.
	  \end{quote}

	  The overall utility of SFL for analysis of online discourse socialisation may be divided into three main factors: the treatment of language as constitutive of text, the notion of language as a meaning-making resource, and the division of interpersonal and experiential functions within a grammar. These three factors are outlined below.

		 \paragraph{Language as constitutive of context}
		 
			Though context is an increasingly central concern within many branches of linguistics, SFL is notable for the extent to which its theory of context has been articulated and empirically applied \cite{widdowson_text_2008}. Halliday explains that in SFL, context and text are in fact seen as``aspects of the same process'', and that a text thus in fact ``goes beyond what is said and written: it includes [...] the total environment in which a text unfolds'' \textcite[p.~5]{halliday_language_1989}.

			Equally important that SFL treats language as constitutive of, rather than simply bound to, its context. As context can often be accurately deduced from text alone, and as context can be used to predict appropriate kinds of texts, some SFL theorists contend that \emph{context is in text} \cite[p.~7]{eggins_introduction_2004}. Context and language thus together construct both a reality and roles for people within it \cite{veel_learning_1997}.

			This notion has often apparently formed an unarticulated theoretical assumption in online discourse socialisation research: indeed, many researchers tacitly take the perspective that language \emph{must} be responsible for the development of distinct cultures in online groups, as it often the sole semiotic system available to group members \cite{thorne_computer-mediated_2008}. The absence of research explicitly drawing upon SFL for online discourse socialisation research is thus striking.

		 \paragraph{Language as a meaning-making resource}

			 The systemic-functional approach foremost involves an understanding of language as a semiotic system strategically drawn upon by language users as a meaning-making resource---``how people use language with each other in accomplishing everyday social life'' \cite[p.~2]{eggins_introduction_2004}. Thus, language in use is \emph{purposive}: its use is motivated by purposes that may or may not be transparent or tangible. This is an appropriate theoretical stance for investigations of online communities, as they are inherently social, and language is the main resource upon which members can draw to achieve their respective goals.

		 %\paragraph{Quantifiability of linguistic features}

			%Halliday explains that it is possible to evaluate texts by counting the frequencies of relevant phenomena...

			%Halliday remarks on the potential usefulness of very large datasets \citeyear{halliday_language_1992}

			%This is amenable to the corpus-based approach taken here.

		 \paragraph{Interpersonal and experiential functions of language}

			From a systemic perspective, the simultaneous provision of health information and social support within OSGs is realised by users attending simultaneously to two metafunctions of language: an interpersonal dimension, responsible for negotiating role relationships, and an experiential dimension, responsible for communicating propositions about the world \cite{butt_using_2009}.

			This is the most useful affordance of SFL for investigation of OSGs, as OSGs exist for the purpose of simultaneous provision of health information and social support: if the social element were not desired, users would likely be content to simply read from static pages; if experiential content were not desired, communities and the threads within them would perhaps not be organised by topic. The emerging interest in the phenomenon of \emph{advice} would likewise greatly benefit from delineation of the relationship between social roles and propositional content which must be embedded within any social transmission of information.

			Sections \ref{par:mood} and \ref{par:trans}) provide an overview of the grammar provided by SFL for these metafunctions.

	  \subsubsection{Overview of relevant elements of systemic functional grammar}

			%\begin{figure}[!ht]
			%\centering
			%\includegraphics[scale=.8]{../../images/neweggins.jpg}
			%\caption[Overview of lexicogrammar, discourse-semantics and context]{Overview of lexicogrammar, discourse-semantics and context (from Eggins, 2004)}
			%%THIS STILL HAS IDEOLOGY...
			%\label{fig:eggins}
			%\end{figure}

		 %\paragraph{Lexicogrammar and discourse-semantics: three metafunctions}

			%Each text occurs within both a \emph{context of situation} and a \emph{context of culture}. The context of situation refers to the configuration of the environment of a specific interaction

			In SFL, language users simultaneously attend to three metafunctions, each of which is responsible for making different kinds of meanings. The \emph{interpersonal metafunction}, realised by the mood system, conveys and negotiates role-relationships between interlocutors. The \emph{experiential metafunction}\endnote{In Hallidayan SFL, experiential meanings and \emph{logical meanings} together comprise the experiential metafunction. In the vein of \textcite{eggins_introduction_2004}, this thesis discusses only experiential meaning, and all references to the transitivity system concern only experiential function.}, realised by the system of transitivity, conveys propositions about the world. The \emph{textual metafunction}\endnote{Despite their potential relevance as a means of explicating the ways in which turn-taking is operationalised in OSGs, textual meanings are not covered in significant detail here due to limitations in scope. Furthermore, the majority of turn-taking oriented research into OSGs has used the better established CA terminological set. This is advocated in \textcite{eggins_analysing_2004}.}, realised by grammatical theme, creates coherence within and between texts. Each function is operationalised at the \glslink{lexicogrammar}{lexicogrammatical} level of language, and as such, can be captured by the systemic functional grammar.

			   %\paragraph{Register: field, tenor and mode}

	  The sum total of the content of each metafunction respectively corresponds to the \emph{tenor}, \emph{field} and \emph{mode} of a text---the three major elements of \emph{register}, which accounts for the situational factors of a text \cite{halliday_language_1989}. Halliday provides a minimal definition of each component of register:

			   \begin{enumerate}
			   \item  ``The \sctext{field of discourse} refers \emph{to what is happening}''
			   \item ``The \sctext{tenor of discourse} refers to \emph{who is taking part}''
			   \item ``The \sctext{mode of discourse} refers to \emph{what part the language is playing}'' \cite[p.~12]{halliday_language_1989}
			   \end{enumerate}

	  Systemic analyses by convention provide a short description of text under each heading (see Chapter 5 for a situational description of the bipolar forum.). The function of these descriptions is ``to interpret the social context of a text, the environment in which meanings are being exchanged'' \cite[p.~12]{halliday_language_1989}. Register is thus the most abstracted level of the \emph{context of situation}---that is, the particular configuration register variables, and the specific environment in which a given text was produced \cite[p.~30]{eggins_introduction_2004}. 

	  Broader than register is the \emph{context of culture}---``the broader background in which the text has to be interpreted'' \cite[p.~46]{halliday_language_1989}. The context of culture may subdivided into \emph{genre}---the  myriad configurations of field, tenor and mode that are recognisable within a culture---and \emph{ideology}---the underlying similarities between all texts produced within a given social group \cite{eggins_introduction_2004,fairclough_language_2001}\endnote{Some systemicists \textcite<e.g.>{martin_genre_2006} have since abandoned the treatment of \emph{ideology}.}.

	  Of most interest to this thesis are the grammar for the mood and transitivity systems, as well as the notion of genre and genre analysis. Each is explicated in more depth below.

		 \paragraph{Interpersonal meanings and the system of mood} \label{par:mood}

			According to the systemic functional grammar \textcite<as outlined in>{eggins_introduction_2004} and \textcite{butt_using_2009}, at the broadest level, utterances involve two potential \emph{speech roles}: \emph{giving} and \emph{demanding}. Within either speech role, two types of commodities may potentially be given or demanded: \emph{information}, or \emph{goods and services}. This leads to four main \emph{speech functions} (See Table \ref{tab:roles}), each of which is congruently realised by a different \emph{mood}:

			\begin{table}
			\centering\small
			\begin{tabular}{|l|l|l|}
			\hline
			\textbf{Speech role} & \textbf{Information} & \textbf{Goods and services} \\ \hline
			\textbf{Giving}	  & statement   & offer			  \\ \hline
			\textbf{Demanding}   & question	& command			\\ \hline
			\end{tabular}
			\caption{Speech roles, commodities and speech functions in SFL}
			\label{tab:roles}
			\end{table}			

			\begin{itemize}
			\item Statements are typically realised by declaratives.
			\item Questions are typically realised by interrogatives.
			\item Commands are typically realised by imperatives.
			\item Offers are typically realised by modulated interrogatives.
			\end{itemize}
			
		 \paragraph{Appraisal as an extension of the mood system}

			SFL inspired the later development of a tripartite model of \emph{appraisal}---that is, the way in which things are described, graded, etc.

			%1. ‘affect’ – the means by which writers/speakers positively or negatively evaluate the entities, happenings and states-of-affairs with which their texts are concerned. it addresses not only the means by which speakers/writers overtly encode what they present as their own attitudes but also those means by which they more indirectly activate evaluative stances and position readers/listeners to supply their own assessments. These attitudinal evaluations are of interest not only because they reveal the speaker’s/writer’s feelings and values but also because their expression can be related to the speaker’s/writer’s status or author- ity as construed by the text, and because they operate rhetorically to construct relations of alignment and rapport between the writer/ speaker and actual or potential respondents 

			%2.Our concern is also with what has traditionally been dealt with under the heading of ‘modality’ and particularly under the headings of ‘epis- temic modality’ and ‘evidentiality’. We extend traditional accounts by attending not only to issues of speaker/writer certainty, commitment and knowledge but also to questions of how the textual voice positions itself with respect to other voices and other positions. In our account, these meanings are seen to provide speakers and writers with the means to present themselves as recognising, answering, ignoring, challenging, rejecting, fending off, anticipating or accommodating actual or poten- tial interlocutors and the value positions they represent.

			%3. We also attend to what has been dealt with under headings such as ‘intensification’ and ‘vague language’, providing a framework for describ- ing how speakers/writers increase and decrease the force of their asser- tions and how they sharpen or blur the semantic categorisations with which they operate.

			%Reports of one’s own emotional reactions are highly personalising. They invite the addressee to respond on a personal level, to empathise, sym- pathise or at least to see the emotion as warranted or understandable. In this, the two letter writers employ a similar intersubjective strategy. The similarity, however, is a relatively fleeting one. The second corre- spondent differs from the first in that, while starting with emotion, he then goes on to provide a number of specific, sometimes technical assessments in support of his viewpoint. Unlike the first writer, he con- structs his role as being, not that of the fan, but that of the expert who would set himself up as the equal of the magazine’s writers and other reviewers.

			%Appraisal is discourse semantic, p/10

			%the better you know someone the more you contract p31

			%fitting appraisal into sfl: At this level it co-articulates interpersonal meaning with two other systems – negotiation and involvement. Negotiation complements appraisal by focussing on the interactive aspects of discourse, speech function and exchange structure (as presented in Martin 1992b). Eggins & Slade 1997 present a detailed SFL framework for analysing interactive moves in casual conversation. Involvement complements appraisal by focussing on non-gradable resources for negotiating tenor relations, especially solidarity. 

			%Attitude is concerned with our feelings, including emotional reactions, judgements of behaviour and evaluation of things. Engagement deals with sourcing attitudes and the play of voices around opinions in dis- course. Graduation attends to grading phenomena whereby feelings are amplified and categories blurred.

		 \paragraph{Experiential meanings and the system of transitivity} \label{par:trans}

		 %\cite{benson_field_1981} section 5 discusses difference between topic and field, and indeed, there is a reason sfl practitioners have been reluctant to use the t word.

		 %The same subject or topic may occur in different fields. It is the semiotic system signalled by the lexis in a text which makes the difference between subject or topic and field. The same event may give rise to similar topics, but the fields of the texts in which these topics are expressed will be determined not by the event but by the semiotic systems of the individuals discussing the event.
   %Take, for example, the topic of wine tasting. The event is clear enough: wine is being tasted and evaluated. Many people, no doubt, would be hard put to produce text, at least extended text, on the topic of their having tasted a particular wine. The condition for text to be produced is that the experience must be given an institutional focus. The same basic phenomenon, tasting and evaluating wine, has given rise to the following two texts, but although their topic is common, their fields are very different \cite[p.~51]{benson_field_1981} 

   %So, the topic is the same as medical journal discussion of bipolar disorder treatment, but the field is differnet due to the presence of different participants and processes, or at least a different lexis for their description.

   %I don't like this analysis much. It seems to me that it is the same field and same topic, but interpersonal, appraisal stuff is at work. Topic modelling works through nothing but lexis and collocation...

   %Topic is a useful term for 'bare' events, but events themselves do not pro- duce language. Text is generated when events are perceived with institutional focus, and that is why we find it useful to speak offield,rather than topic, in a text. An apparent exception to this is the phenomenon of 'phatic communion'. Casual discourse at a cocktail party, between acquaintances waiting in a queue, and so forth, will tend to range from topic to topic without, for the most part, displaying institutional focus. What is happening here is that the institutional focuses negotiated during such encounters are so slight that the resultant fields are barely perceptible.

			The second kind of meaning to which speakers attend is the \emph{experiential metafunction}. As clauses in SFL are multifunctional, experiential and interpersonal meanings are expressed simultaneously, but through different subsystems within the grammar. For each clause, experiential meaning  involves the instantiation of a \emph{verbal process type} within the verbal group. Each process type constrains the available configurations of participants (nominal groups) and circumstances (adverbial or prepositional phrases). Circumstances, like residue, have their own sub-grammar. Various tests can be used to disambiguate more difficult cases. In the sections below, I sketch a simplified version of the `Sydney School' experiential grammar---a fuller account can be found in \cite[pp.~220--270]{eggins_introduction_2004}.

			   \cleardoublepage
			   \begin{table}[!ht] \small{
			   \begin{tabularx}\linewidth{|L{0.10}|L{0.50}|L{0.50}|L{0.50}|L{0.50}|L{0.50}|X|}
			   \hline
			   \textbf{\#} & \textbf{Process type} & \textbf{Definition} & \textbf{Typical verbs} & \textbf{Identification test} & \textbf{Participants} \\ \hline
			   1.  & Material							& Doing tangible things								 & \emph{kick, run, draw}		 & \emph{What did x do?}						  & Actor (goal) (range) (beneficiary) \\ \hline
			   2.  & Mental							  & Thinking or feeling								   & \emph{think, believe}		  & \emph{What do you think, feel or know about y?}					 & Senser, phenomenon						  \\ \hline
			   3.  & Verbal							  & Saying												& \emph{shout, yell, tell}	   & \emph{What did x say?}						 & Sayer, (receiver) (verbiage)		 \\ \hline
			   4.  & Behavioural						 & Psychological and \newline physiological behaviour	& \emph{cough, smile, dream}	 & Unmarked present tense has continuous sense											 & Behaver (behaviour) (phenomenon)	   \\ \hline
			   5.  & Existential						 & `There' clauses									   & \emph{be }					 & Presence of non-locative \emph{there}		  & Existent							\\ \hline
			   6a. & Relational \newline (identifying)   & Ways of being										 & \emph{equal, mean, symbolise}  &  \emph{x is a member of the class a.}		  & Attribute, carrier				  \\ \hline
			   6b. & Relational \newline (attributive)   & Defining											  & \emph{own}					 & \emph{x serves to define the identity of y.}   & Cannot be passivised				 \\ \hline
			   \end{tabularx} 
			   }
			   \caption{Summary of verb process types}
		   \end{table}

			%\begin{figure}[!ht]
			%\centering
			%\includegraphics[width=3in]{../../images/circ}
			%\caption[System of circumstance]{System of circumstance (from Eggins, 2004)}
			%\label{fig:circ}
			%\end{figure}

			Of particular interest in this thesis are the kinds of participants, which are mapped to heads of NPs
		 %\paragraph{Theme: textual metafunction} \label{par:theme}

			%``The Theme is the element which serves as the point of departure for the message, it is that with which the clause is concerned' or again `the Theme is the starting-point for the message; it is what the clause is going to be about'''. \cite[p.~158]{huddleston_constituency_1988}

		 %``In English, Halliday says, the Theme is identifiable as `that element which comes in initial position in the clause''' \cite[p.~158]{huddleston_constituency_1988}
		 
		 % paragraph theme_textual_metafunction (end)

		 \subsubsection{Context of culture: genre} \label{genre}

			\noindent In contrast to the context of situation, which concerns the environment in which a given text is produced, the context of culture refers to the broader conditions common to texts. Culture is the highest level of abstraction in SFL---all other meaning systems exist within and belong to cultures \cite{halliday_language_1989}. Thus, the context of culture is simply the semiotic potential of the totality of sign systems. 

			The best-articulated component of the context of culture is the notion of \emph{genre}---that is, ``recurrent configuration[s] of meaning, phased in discourse as a staged, goal-oriented social process'' \cite[p.~9]{martin_genre-based_2013}. Genres ultimately derive their meaning from their instantiation within a given culture \cite[p.~99]{halliday_language_1989}. Though genre and register alike are realised by the contextual variables of field, tenor and mode, genre theory emphasises the social purpose of the activity being undertaken in the text \cite{woodward-kron_disciplinary_2002}. Thus, genres may be characterised as constellations of field, tenor and mode that are culturally recognised as performing social functions. It is also important to bear in mind when considering genre that individual genres are not necessarily distinct: \emph{macro-genres} may contain \emph{micro-genres} of genre stages derived from micro-genres: the macro-genre of the \emph{essay}, for instance may draw on micro-genres of \emph{expositions}, \emph{discussions} and \emph{evaluative accounts} \cite{woodward-kron_disciplinary_2002}.

			~\\\todo[inline,color=green!40]{I need to expand upon the micro-and macro-genre issue, as it is very relevant to the investigations: first posts are seen as a micro-genre within a meso-genre of first post threads; threads themselves form a macro-genre.}

			\textcite{eggins_analysing_2004} provide simplified parameters for performing genre analysis. The six steps are outlined below. Unless otherwise specified, the reference for the following sections is \textcite{eggins_analysing_2004}

			\paragraph{Recognising a generic text}

			Given that genres are culturally recognised by their very nature, simple reading of texts by those fluent in the relevant culture may be enough to identify the existence of a genre. \citeyear[p.~213]{eggins_analysing_2004} explain that genres are recognised when texts appear to ``move through predictable stages''. \textcite{eggins_introduction_2004} argues that the existence of a word for the kind of behaviour seen in the text (i.e. \emph{purchasing}, \emph{commentating}, \emph{gossiping}) may at the very least be a clue that a potentially definable genre exists.

			%It must be borne in mind when attempting to locate genres that there 

			\paragraph{Defining the social purpose of the generic text}

			   This step involves clarifying the overall function(s) of a genre with as much specificity as possible---rather than ``telling a story'', sub-categorisations such as ``anecdote'' or ``exemplum'' are more useful, given the existence of micro- and macro-genres. More theoretically, this task also involves developing an understanding of how the genre constructs a social reality: by virtue of their existence alone, recognisable genre stages may provide insight into social practices and culturally accepted attitudes and values. 

			\paragraph{Identifying and differentiating stages within a genre}

			   After breaking down the text into clauses as with \glslink{lexicogrammar}{lexicogrammatical} analysis, groups of these clauses must be divided by their role within the text. These roles should be functionally, rather than formally defined: \emph{Abstract or Resolution} is superior to \emph{Beginning} or \emph{Chapter Three} because the latter are not genre specific. By convention, each of these functions should then in turn be described in prose.

			\paragraph{Specifying obligatory, optional and recursive stages}

			   Stages may be obligatory, optional or recursive. Obligatory elements are considered to be defining features, and in some cases may be unique to the genre under investigation. Optional stages, on the other hand, are likely to be present in other genres. \textcite{halliday_language_1989} remind us that optional stages do not occur randomly: in the \emph{buying and selling genre}, the number of customers in the store or the size of the line may affect whether or not a \emph{greeting} or \emph{sale initiation} takes place. Both optional and obligatory stages may be recursive, as in the case of the buying and selling genre, in which \emph{sale request}, \emph{sale enquiry}, \emph{sale compliance} and \emph{sale} may go through limitless iterations \textcite[p.~61]{halliday_language_1989}.

			\paragraph{Devising a structural formula}

			   The next task is to represent the genre structure. By convention, each genre stage is delineated by a carat (~$\hat{}$~). Optional stages are bracketed. Recursive stages are square-bracketed, with brackets followed by  \textsuperscript{n}.

			   A number of competing methods of representing the relationship between genre stages have been proposed, and many have undergone revisions in order to be easier to render on a computer \cite{eggins_introduction_2004}. Problematic is that many \cite<e.g. those in >{halliday_language_1989} are esoteric, and lagging behind the representational schemes of other grammars in terms of readability \cite{hovy_types_1996}. In fact, the development and use of unique means of expressing optionality and recursion is perhaps superfluous and ultimately unhelpful, given that comparatively well-known systems such as \emph{regular expressions} could potentially represent generic structure with ease.

			   A particularly important addition to structural representation of genre is Hasan's \citeyear{hasan_structure_1985} notion of \emph{generic structure potential}: that is, a maximally expanded representation of genre staging that exhausts all possibilities for additional optional stages and recursion\endnote{It is important to distinguish generic structure potential from \emph{genre potential}---that is, ``all the linguistically-achieved activity types recognised as meaningful (i.e. appropriate) in a given culture'' \cite[p.~35]{eggins_introduction_2004}. In practice, this translates to every possible configuration of field, tenor and mode.}.

			\paragraph{Analysing the semantic and \glslink{lexicogrammar}{lexicogrammatical} features for each stage of a genre}

			As Hasan explains,

			\begin{quote}\singlespacing\small
			a text has many modes of existence and so it can be analysed at many different levels, with each contributing to our understanding of the phenomena involved \citeyear[p.~116]{halliday_language_1989}.
			\end{quote}

			\noindent Thus, relevant parts of systemic functional grammar may be operationalised in order to investigate the phenomena of interest: a researcher interested in power dynamics within a text, for example, would likely perform an analysis of mood features, as these are responsible for the management of role relationships. Simultaneously, this analysis provides a justification of the treatment of the text as instantiating a genre and of the clause complexes as instantiating generic stages. \textcite{eggins_analysing_2004} note that since lexicogrammar can provide hints as to genre staging, this step of the analysis may render it necessary to reconsider the previous delineation of stages.

			Genre may influence the lexicogrammar of texts in two ways. First, as genres are configurations of register variables, texts within genres must necessarily conform at the level of lexicogrammar. In this way, a genre such as \emph{sports commentary} is likely to bring about language which experientially positions players, teams, umpires and coaches as the main participants. Second, different genre \emph{stages} may influence \glslink{lexicogrammar}{lexicogrammatical} decisions. The \emph{evaluation} stage within the \emph{storytelling} macro-genre, for example, is likely to opt for a declarative mood, while experientially, it can be assumed that mental and relational process types may occur.

	  \subsubsection{Criticism of systemic functional linguistics}

		 \textcite{van_dijk_text_2004} has criticised three main dimensions of SFL. First and most broadly he notes that its sheer density creates difficulty when a researcher seeks to use only relevant sections of the theory:  ``not only are the terms (field, tenor, mode) hardly transparent, as to their intended meanings, but also the usual---informal---descriptions of their meanings are barely enlightening'' \citeyear[p.~341]{van_dijk_text_2004}. Second, he characterises SFL as having lacked sufficient engagement with potentially useful interdisciplinary perspectives: ``there is very little inspiration from the many other approaches to context in linguistics and especially in anthropology, sociology or social psychology, at least in the analysis of the context'' \citeyear[p.~342]{van_dijk_text_2004}. Finally, he points out that SFL has largely avoided cognitivist accounts of language. In fact, given the amount of work focussed on machine text generation, some SFL theorists are attracted to the theory precisely due to its lack of consideration of the role of the mind in language production \cite{odonnell_[sys-func]_2014}

		 %Although the contexts of the three categories of context of situation as formulated in SFL are slightly different for different authors, the notions have not changed much in the last 30 years. Theoretically the notions are rather vague and heterogeneous, and it is striking that for a functional theory of language that aims to provide a 'social semiotic, context structures have not been explored more systematically and more explicitly in all these years. Not only are the terms (field, tenor, mode) hardly transparent, as to their intended meanings, but also the usual — informal — descriptions of their meanings are barely enlightening. \cite{van_dijk_text_2004}. Van Dijk also criticises the lack of inpsiration from other linguistic approaches to context, despite many having rich and successful histories.

		 \textcite{huddleston_constituency_1988}  and \textcite{widdowson_text_2008} have independently criticised SFL's conceptualisation of thematic metafunction in particular:  both argue that the notion of theme as simply the first element in a clause is problematic (in cases where there are dummy subjects, etc.). Van Dijk has further criticised what he sees as an attempt to force \glslink{lexicogrammar}{lexicogrammatical} \emph{conjunction} to line up with the register variable of \emph{mode}. Perhaps this criticism is tacitly embraced by \textcite{eggins_analysing_2004}, who tend to advocate the use of terminology and concepts from conversation analysis, rather than SFL, for investigations of turn-taking.  For these reasons, compounded simply by issues of scope, I have opted not to consider textual meanings in this thesis in any serious detail. This is perhaps unfortunate, as much could foreseeably be learned about OSGs through analysis of the ways in which the take-up of advice (for example) is realised in forum threads.

		 %Van Dijk's comment that SFL is no longer unified, rather than being taken as a criticism, is taken as a virtue: the partial use of relevant parts of the grammar allows focus, etc.

	\subsection{Corpus linguistics}

		Corpus linguistics is presented as an addition to research into online discourse socialisation and CMDA. In this section, I outline relevant practices and literature. Key areas of interest are corpus-assisted discourse research and web-corpus based approaches. Due to the size and complexity of corpus linguistics as a research area, explanation of key terms is provided in the glossary. Shortcomings in current theory and practice are highlighted.

		\subsubsection{Current theory and practice}

			Essentially, corpus linguistics involves three tasks:

			\begin{enumerate}
				\item the transformation of natural language data into forms that can be analysed using computational tools (corpus building)
				\item the quantitative interrogation of these datasets (corpus interrogation)
				\item The development of tools to accomplish 1 and 2 (tool creation)
			\end{enumerate}

			In some investigations, individual researchers perform all three tasks. In others, interdisciplinary teams are formed. In others still, existing tools and resources are used...

		\subsubsection{Corpus assisted discourse studies}

			Discourse is an increasingly common focus of corpus linguistics.

			A number of overlapping and competing names for the research area have been used, each attempting to convey the extent to which corpora are relied upon in order to answer the research question.

			%is it lee or flowerdew or someone who does into this?

			\begin{enumerate}
				\item Corpus-based ...
				\item Corpus-informed ...
				\item Corpus-driven ...
				\item Corpus-assisted ...
			\end{enumerate}

			\begin{itemize}
				\item List of studies and their methods...
			\end{itemize}

			\paragraph{Shortcomings in CADS}

				\begin{itemize}
					\item available technology is prohibitive...
					\item Corpus building tools are scarce
					\item Corpus-sharing is difficult
					\item researchers are typically not trained in CS
					\item Common practices are not really nuanced---collocation etc.
				\end{itemize}

		\subsubsection{Web corpora}







%\bibliography{../references/libwin.bib}
