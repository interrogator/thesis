%!TEX root = ../thesis.tex

% critically examine your findings in the light of the previous state of the subject as outlined in the background, and make judgments as to what has been learnt in your work”

\chapter{Discussion} \label{chap:discussion}

    In this chapter, I relate these findings more directly to the first three research questions posed in Chapter \ref{chap:intro}:
    
    % to the systemic notion of register, to literature reviewed in Chapters \ref{chap:onlinehealth} and \ref{chap:approaches}, and then, finally, 
    %In this chapter, I focus on the register used by Forum members, and how this shifts over the course of membership. This is then linked back to literature reviewed in Chapter \ref{chap:onlinehealth}, with discussion of the strengths and weaknesses of legitimacy and socialisation as ways of understanding linguistic change in \glspl{OSG}.
    
    
    %The final part of this chapter organises a discussion of key findings as they relate to the four research questions posed in Chapter \ref{chap:intro}:
    
    \begin{enumerate}\setlength\itemsep{0em}
\item Which components of \gls{lexicogrammar} are \emph{at risk}\slash subject to change over the course of membership within an \glsxtrlong{OSG}?
\item How do these changes relate to \glspl{discourse-semantic} and register in the \gls{OSG}?
\item What implications do the findings generated by this case study have for corpus linguistics, corpus\hyp{}assisted discourse studies, systemic\hyp{}functional linguistics, and healthcare communication research?
\item What kinds of tools and mthods are needed in order to effectively analyse the data, and, more generally, to perform functionally driven analysis of natural language \glspl{corpus}?
\end{enumerate}

    %
    The final research question, concerning the implications of the case study, is addressed separately, in Chapter n.
    





\begin{comment}


% Methodologically, these features are the ones that make \glspl{OSG} a potentially valuable data source for healthcare communication research. The lack of constraints on what can be written mean that large amounts of content are authored on diverse subjects.



%The complementarity of interpersonal and experiential meanings makes rich insights into linguistic phenomena possible. In the case of \emph{I would} advice, the veteran can reconstrue him\slash herself as the Agent in the experiential scene, while simultaneously providing information and incongruently demanding action on the interpersonal plane.

%The first area of discussion is the relationship between \gls{CMC} and online community literature and the \glslink{Forum}{Bipolar Forum}. Like other \glspl{OSG}, the \gls{Forum} is a prototypical example of an \glslink{forum}{online forum}. In turn, these \glspl{forum} are longstanding parts of the overall landscape of \gls{CMC}---though as mentioned earlier, there is evidence to suggest that many (including the \glslink{Forum}{Bipolar Forum}) are now in a state of decline.  The Bipolar Forum is powered by vBulletin, a common framework for message boards online, leading to a great deal of similarity in terms of the textual dimension between countless other online communities, and thus, 

%Analysis of linguistic content of members' posts aligns with what has generally been noted in more recent online community literature. Users take on distinct roles within the community that may shift with experience. The lack of a moderator of group leader means that the task of maintaining productive and coherent dialogue is managed by the veteran members.

%Another point of commonality between online community literature and the findings of the case study is that the community does indeed serve the dual functions of interpersonal and ideational exchange: users

%In \gls{SFL}, texts are understood as necessarily and simultaneously creating meanings of both kinds.

% negative behaviour

%\textcite{crowston_reproduced_2000} argue that modes of CMC can be conceptualised according to their relationship, or lack thereof, with offline modes: some types of \gls{CMC} replicate an offline counterpart, while others reconfigure it. Others still do not have an observable antecedent. A key question is whether or not such a scheme can usefully be applied to the Bipolar Forum, or to \glspl{OSG} in general. One possibility would be to consider \glspl{OSG} as reconfigurations of face-to-face support communities such as Alcoholics Anonymous.

%Because the shift from face-to-face to online support groups is essentially a shift in Mode, it is predominantly textual meanings that are at risk, and which form the main point of difference between the two text types. Though I did not perform a lexicogrammatical analysis of the system of \sctext{Theme}, we can still quite easily note the ways in which the \gls{OSG} has reconfigured the Mode of Discourse. The ability to come and go with little investment, to remain anonymous, and to speak candidly without facing lasting social sanctions result in the constant influx of new members, nearly all of whom do not maintain a long-term presence in the group. The few veteran members do a disproportionate amount of 

%Over the course of the case study, no evidence was found for overtly hostile or negative social behaviour. This aligns with more recent \gls{CMC} research, which has downplayed earlier descriptions of anti-social behaviour online. This is not to say that such behaviours are uncommon in contemporary \gls{CMC} more generally. 

%Platforms and sites that attract opposing viewpoints on contentious issues are more likely to contain such hostile interactions,

%This is in stark contrast to the cascommunities in which members are united against a common problem, and in which users may derive health benefits from positive interpersonal exchange.

% \textcite{walther_computer-mediated_1996} --- deep social relationships can develop


Shifting away from the online forum Mode, however, it becomes more difficult to claim that findings may be generalisable to other parts of the landscape of \gls{CMC}. The single\hyp{}threaded forum mode is increasingly distinct from that of many popular text\hyp{}based CMC \glspl{mode} in Web 2.0, which allow users to respond to individual posts or comments within individual threads (Facebook wall posts, Reddit threads, and many news commenting platforms are good current examples). These kinds of architectures create tree\hyp{}like interaction structures, where later contributions branch off from early ones recursively, facilitating easily distinguishable sub-discussions within a broader topic or thread. In the Bipolar Forum, like many others, this kind of multi-threadedness is not encouraged by the site architecture: threads are read from earlier to most recent contribution; users must manually flag the fact that they are addressing only one specific post by commenting or using vocatives. It becomes very difficult, therefore, to generalise 

As seen in the qualitatively analysed examples of Chapter \ref{chap:introdata}, however, contributors to a thread may address only the content of the original post, ignoring the contributions of others. For this reason, similar kinds of advice are repeated to a newcomer.

even if community members still adopt

The same problems exist at a methodological level: while features of the Forum that make it amenable to computational\slash corpus linguistic analysis are often not found in other, perhaps more popular \glspl{mode}.


Because there are no limits on post length, and because interactions do not necessarily need to unfold in real-time, contributors are able to compose complete messages, and potentially edit for spelling and grammar before

This level of formality makes it possible to use automatic annotation and parsing with some degree of accuracy. For other \glspl{mode}, such as Twitter or YouTube, parsing is more difficult, due to greater departures from the standard written English on which parsers are generally trained.







The large number of members and contributions to the \gls{Forum}, as well as its ten-year period of popularity, increase the likelihood that a number of key findings may be generalisable to similar communities and \gls{CMC} \glspl{mode}.



% Web 2.0?

%Though \glspl{OSG} are named to reflect offline antecedents, so too are forums and bulletin boards, 

\subsection{Online community}


Low barriers to entry lead to a community that generally has many new members and very few veteran members, and as such, many first posts that receive replies from users at all membership stages. 



More difficult to assess from this case study is the extent to which the Forum can be characterised as a \emph{community}---indeed, the broadening of this term to include thousands of participants, many of whom have never directly addressed one another, has been criticised. The difficulty lies in the fact that the term has no single definition. In terms of Communities of Practice theory, it seems that the Forum meets the criteria of a shared enterprise, and, at least for a portion of members (especially veterans), a distinct repertoire of linguistic practices that range from orthographic\slash lexical choices of abbreviations and jargon terms to ideological norms concerning the 

Joint endeavour and shared repertoire can be understood in systemic terms as the two edges of the hierarchy of stratification, with shared repertoire consisting of lexicogrammatical and \gls{discourse-semantic} phenomena and joint endeavour consisting of the registerial and ideological values that these phenomena realise.

% maybe a figure showing this?

% the problem is that shared goal may not be shared ideology

There is a great deal of overlap between the notion of a shared enterprise\slash goal and a shared ideology. In the case of the \glslink{Forum}{Bipolar Forum}, it is not possible to distinguish between the normative goal of managing bipolar disorder and the normative ideology that frames bipolar as manageable through a combination of proper treatment from a professional and a continued effort by the afflicted to understand the condition and minimise the negative consequences of manic and depressive parts of the cycle.



\section{New members}

Most linguistic accounts of \glspl{OSG} have focussed on new members' initial contributions. The case study adds to this body of literature, largely supporting existing claims. First, findings from both Mood and Transitivity analyses support the assumption\slash assertion that new users are in a position of little authority, compared to veterans: commands are rare, and requests often involve one or more of many politeness strategies.

% mood finding

Though \sctext{Mood} is the system responsible for negotiating role-relationships, other linguistic features may play a role. Shallow features such as average word length and the proportion of all words that are nouns rise steadily over the course of membership, indicating a register that is in some senses scientised and increasingly dense.

% shallow finding
% transitivity finding

This metafunctional interplay is discussed in more detail in the following chapter.

Also supported is the notion that first \glspl{post} can conform to a generic structure.
% who said?
This structure is similar to that noted in other projects.
% what is it?








\section{Legitimation}

As demonstrated by a number of studies reviewed in Chapter \ref{chap:onlinehealth}, legitimacy is a powerful way of understanding many of the linguistic choices made by both new and veteran members in \glspl{OSG}.

New members, for example, place great emphasis on the process of diagnosis---it is the most key of all processes in first posts. Those who have been diagnosed point out its recency through the use of temporal circumstances; new users without diagnoses provide evidence for the likelihood of their having bipolar regardless. This is done because having a diagnosis indicates a legitimate reason for joining the community; 


Veteran members stress the validity of diagnosis as a way of checking tickets. 

\subsection{Shortcomings of the legitimacy framework}

One issue encountered with the notion of legitimacy in the context of \gls{OSG} analysis is the fact that opposite behaviours can be convincingly be analysed as legitimation strategies. A user can highlight his\slash her prior research into the problem he\slash she currently faces, lurking in the community and participation in similar groups in order to represent a Self that is deserving of replies. Conversely, a user can demonstrate a legitimate need for help by communicating a crisis and a lack of understanding of important issues.  With this in mind, it may be helpful to cluster initial contributions into sets of strategies that appeal to different motivations for issuing a reply.

Urgency, noted as a useful strategy by \textcite{horne_doing_2009}, occupies one end.

Contentiousness (trolling, in some CMC modes) may also provoke responses.

Politeness and deference

Offering contributions that may be of personal value to others

Further research would do well to examine the kinds of responses elicited by the different kinds of legitimation strategies selected by incoming members.

\subsection{Acting new}

% note little evidence for whether people have lurked

\section{Veteran members}

Many earlier findings regarding the language use of established online community members were supported by the case study.

Functions such as welcoming, advising, commanding become more common, while thanking becomes less common.

Jargonisation emerges, as well as more delicate distinctions between important components of the field of discourse, including kinds of health professional, medications and symptoms.



This increasing preference for jargonisation has been noted in a number of studies of online communities \cite{danescu-niculescu-mizil_no_2013}.

Veterans also use metadiscourse to describe to others the purpose of the community itself.

Vague language?

%`Part of the purpose of this forum is for venting'


Veteran members construe participants in a way that aligns more closely with the ideology of the Forum: health professionals feature more prominently as Agents in veteran talk, as does the Self. Alignment with dominant ideologies may signify authority and social power.











%\subsection{A qualitative understanding of contributions to the Forum}

%At this point, we can create a qualitative sketch of interactions in the \gls{Forum} at differing stages of membership. The generalisation of this thread to others is informed by the quantitative analyses performed in the next two chapters. 







\subsection{Legitimacy}

Veteran members also use language in a way that highlights legitimacy, in order to increase the perlocutionary force of suggested future behaviour for the addressee. As previous research into legitimation strategies has focussed on those in positions of power, a pre-existing theoretical framework for classifying strategies is available \cite{van_leeuwen_legitimation_2007,reyes_strategies_2011}. The framework, however, lacks definitive statements of how legitimation is realised in lexicogrammar. Nevertheless, it is possible to identify how the major strategies (x, y and z) map to instantiated linguistic choices in the \gls{Forum}:

\begin{enumerate}
    \item 
    \item 
    \item 
    \item 
\end{enumerate}
%
Because there is less research on legitimation by those with less social status (i.e. new members), it is difficult to apply these categories to the longitudinal dataset.

% ATTEMPT TO DO SO?

A more comprehensive account of legitimation in online communites would map the relationship between shifting linguistic choices and shifts in the kinds of legitimation strategy that are appropriate at each membership stage.

\subsection{Proto-professionalism}












\subsection{Shifting legitimation strategies}

Many attempts have been made to identify discursive legitimation strategies and to map these to components of the \emph{lexicogrammar}. As overviewed in Chapter \ref{chap:onlinehealth}, within a systemic-functional framework, \textcite{van_leeuwen_legitimation_2007} identies four main strategies

\begin{enumerate}
    \item Appeal to 
    \item 
    \item 
    \item 
\end{enumerate}

\textcite[p.~782]{reyes_strategies_2011} augments these to deal with the ways in which speakers legitimate or justify actions:

\begin{enumerate}
    \item Hypothetical future
    \item Rationality
    \item Expertise
    \item Altruism
\end{enumerate}
% The use of emotions
% a hypothetical future
% rationality
% expertise
% altruism

These strategies can be found in veteran \gls{Forum} \glspl{member}' talk: hypothetical futures commonly figure into the provision of advice; at the same time, advice foregrounds the veterans' expertise, having personally experienced the problems facing newcomers. Altruism also permeates veterans' talk: veteran users do not express their desire for influence or control within the community, instead representing themselves as concerned for the well-being of others.

Notably, however, legitimation strategies from the literature do not seem to apply to new users' language use: while documenting the ways in which powerful people peruade others through discourse, less has been done to determine how newcomers frame themselves as warranting responses. In the \glslink{Forum}{Bipolar Forum}, users are more likely to stress a lack of knowledge and need for guidance. Moreover, positioning the self as an expert may be seen as a challenge to the authority of veteran users, and thus, to the normative values of the community.

In fact, it seems that newcomers and veterans opt for opposite representations of the self. Though those without authority also need to legitimate themselves within a community structure, this cannot be accomplished by the construal of the self as an expert. If the newcomer is an expert, his\slash her need for information is perceived to be low; ...



critical to the construction and reinforcement of medical ideologies is the extent to which professionals and consumers are construed as able to act and effect change. and the position of others as simple experiencers of actions, or media through which change occurs. 



\section{Socialisation}

In Chapter \ref{chap:onlinehealth}, I summarised four challenges posed by \gls{CMC}, \glspl{OSG} and \gls{SFL} a socialisation\hyp{}based theory of longitudinal linguistic change. These four challenges are briefly discussed in the sections below.

\subsection{The phenomenon of lurking}

The fact that some contributors lurk extensively before choosing to contribute poses a prblem for a socialisation-based account of longitudinal change, because users may learn a great deal without active participation.

\subsection{Scope and strata of socialisation}

The second issue is whether or not we can precisely define what it is that Forum members are being socialised into.

\subsection{Prior knowledge of community norms}

Socialisation may also struggle to account for 


\subsection{Epistemological issues when using CMC data}

The final problem is whether or not evidence can be provided for socialisation without access to other kinds of data, such as interviews with Forum members.

We can see countless ways in which language changes over the course of membership.

We can also link these to 


\subsection{Combining a top-down and bottom-up approaches to change}




Fortunately, changes themselves are interesting in their own right: regardless of their cause, changes represent effects of involvement in the community on language production, and, more generally, effects of involvement on the illness course. Moreover, in most workflows, computational or otherwise, this kind of auxiliary data is unlikely to be readily or feasibly obtainable. Accordingly, it is important to work with theoretical frameworks that can provide conceptualisations that can be acted upon and empirically tested.








\section{What the Bipolar Forum does}

The roles and goals of veteran members differ substantially from those of new members. Generally, veterans are understood to be in a position of authority, acting as gatekeepers, dispensing advice, and speaking on the behalf of the community. Like newcomers, they seek legitimacy in order to increase perlocutionary force. Also like newcomers, they also request information from other members. This information, however, is generally information from other users' personal histories, rather than facts and figures about health. It is sought in order to inform the provision of more accurate advice.

\subsection{Situating the findings}

Broadly speaking, findings have aligned with those of previous, qualitative investigations of \glspl{OSG}, as well as the smaller body of quantitative and computational studies. The key contributions of the thesis are in demonstrating that qualitative insights can be empirically and quantitatively accounted for, and that their identification can be a largely automatic process---though a degree of contextual sensitivity is of course lost. For quantitative studies, the methods show that sustained engagement with functional linguistic theory can provide richer and deeper insights than are possible through analysis of lexis alone.

The creation of tools and description of a methodological workflow for carrying out these kinds of investigations on novel kinds of data are a further contribution, to be discussed in detail in the following chapters.

\subsubsection{Toward a connection with health outcomes}

The case study of the thesis was designed to show that discursive features in large bodies of text can be semi-automatically located and quantified. With the selected data and developed methods, however, it is not possible to make an explicit link to healthcare outcomes.

Studying \glspl{OSG}, it is possible to make such connections. \textcite{yan2014feeling} exploit users' self reporting of weight loss, in combination with \gls{forum} \glspl{post}, to map forum use to users' goal of weight management. More computationally sophisticated methods also open up \gls{OSG} analysis to health outcomes without the availability of such self-reported data. \textcite{maclean_forum77:_2015} demonstrate that it is possible to train a computational model that distinguishes between stages of drug addiction (using, withdrawal, relapse), and to automatically identify these stages in users' history of forum contributions.

Pathways toward a sustainable connection of natural language about healthcare to clinical outcomes are proposed in Chapter \ref{chap:futuredirections}.








The case study of this thesis involves analysis of one such online community. As such, in addition to being situated within \gls{CMC} research, the study must also n the domain of healthcare, inside both \emph{healthcare communication} \cite[e.g.][]{slade_communicating_2015} and, to a lesser extent, clinically focussed \gls{NLP} \cite[e.g.][]{friedman_natural_2014}. Healthcare communication research spans intra-professional, professional--consumer and, more rarely, intra\hyp{}consumer interactions. It is typically centred on clinical settings, but media and \gls{CMC} are also of interest. The underlying hypothesis is that language use in healthcare contexts is an exploitable resource from which insights for healthcare policy and practice can be produced. Broadly speaking, \emph{healthcare communication} is a part of the emerging\emph{patient\slash relationship centred care model}, where increased attention is paid to the patient journey, with the aim of safer, more effective and compassionate care. This kind of communication correlates with improved treatment outcomes, including more efficient and accurate diagnosis \cite{slade_communicating_2015} and compliance with treatment plans \cite{crane_patient_1997}. The field contrasts with medical \gls{NLP}, which is generally more interested in computational modelling for the purposes of prediction (of needed resources for patient management, of health outcomes, etc). As such, medical \gls{NLP} has more often utilised \gls{CMC} data, which is freely available in quantities sufficient for quantitative work. Both branches of healthcare communication research, however, share an overarching goal of improving the quality of care through a heightened sensitivity to language and communication.




%The metafunction hookup hypothesis conceptualises an interrelatedness of field, tenor and mode variables, whereby changes in one dimension are likely to have ramifications in another. Broadly, 









\subsubsection{The influence of members on Forum culture}



\begin{figure}[htb]
\centering
\includegraphics[width=0.6\textwidth]{../images/longitudinal-ascriptions.png}
\caption{Ascriptions of bipolar disorder, longitudinally}
\label{fig:ascriptions_l}
\end{figure}

Figure \ref{fig:ascriptions_l} tracks ascriptions of \gls{bipolar} longitudinally using the L Corpus. Here, we can note two things. First is that the popularity of the \emph{having} construction correlates with the period dominated by veteran \gls{member} \glspl{post} (see Section \ref{sect:l-corpus}). Second is that the shift appears to become normative: even after the veteran \glspl{member} have stopped contributing, \emph{being} forms continue to be displaced by \emph{having} and \emph{with} forms. There are two potential causes for this. First, because veteran \glspl{member}' \glspl{post} are still accessible to new \glslink{member}{users} (by navigating back through archives, search, etc.), their influence can extend beyond their period as active contributors. Also potentially at work, however, are general shifts in public and media mental health discourse in the early 21st century, characterised by destigmatisation, increased awareness and increased attention on the way the mentally ill are construed in everyday talk. These potential shifts in cultural values in broader social structures than what is represented in a \gls{corpus} may confound diachronic \gls{CL} in general, but in other cases, may in fact constitute the real targets of the corpus analysis \cite{zinn_changing_2015}.

















\end{comment}
