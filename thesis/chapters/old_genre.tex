%!TEX root = ../thesis.tex

\chapter{Generic features of \emph{Bipolar Forum}}

In the previous chapter, I performed a corpus analysis. This chapter contains a contextualised qualitative analysis of a thread.

\section{Genre: a systemic-functional approach}

	\subsection{Genre in CMC}



\section{Summary of content plane features at risk longitudinally in the Bipolar Forum}

\subsection{Mode} % Medium, channel, division of labour, orientation toward field/tenor, 

The two kinds of spaces in which bipolar meanings are communicated are thus:

\begin{enumerate}
    \item The forum, its threads and posts % written, dialogic
    \item Largely non-interactive resources within the website, news, etc % written, 
\end{enumerate}

        \begin{table}[htb]
         \centering
          \addvbuffer[12pt 8pt]{\begin{tabular}{|l|l|l|}
          \hline
          \textbf{Type} & \textbf{Medium} & \textbf{Turn} \\ \hline
          Posts and threads in the forum & Written & Dialogic \\ \hline
          Static resources & Written & Monologic         \\ \hline
        \end{tabular}}
        \caption{Main types of communication, their medium and turn facets}
        \label{tab:types}
        \end{table}
%
That said, members may communicate through private messages, to which we have no access here, or may arrange to communicate over other media, or even face-to-face.

In SF terms, we can say that most prominent

This restriction of possible modes\slash media is common to many CMC environments, though there is a tendancy for multimodality and multi-channel availability in popular newer media, such as Skype. Given that a complete systemic description of an institution is a complex and substantial undertaking, focussing on sites with some key restrictions allows 

\subsubsection{Division of labour}
Language can play a number of different roles within an institution: at one end, language can constitute the institution and its activities; at the other, language may only facilitate or describe the main activities. % The \emph{division of labour} between matter and meanings, IFG 38

In the case of the Bipolar Forum, and CMC sites in general, language plays a constitutive role: it is difficult to imagine that the community could be sustained if it were not possible to create and respond to others' posts.

The strength of the methodology is in simplifying the process of locating \emph{wordings at risk} over the course of membership in the community.

\subsection{Meanings at risk}

Also within the content plane is discourse-semantics. We can thus

Changes can be observed within each metafunction. Furthermore, metafunctions tend to implicate one another

As observed in the previous chapter in the discussion of jargonisation, membership patterns

%\subsubsection{Interpersonal meanings at risk}

\textcite{halliday_introduction_2004} posit that socio-semiotic situations involve both \emph{doing} and \emph{meaning}, often interdependently. This division is similar to the \emph{matter}\slash \emph{meaning} distinction posited in Halliday 2005.

Meanings can be divided into seven distinct types, which can also be typologised:

\begin{enumerate}
    \item Expounding
    \item Exploring
    \item Recommending
    \item Enabling
    %\item Doing
    \item Sharing
    \item Recreating
    \item Reporting
\end{enumerate}
%
These categories are not necessarily discrete, however. 
%
We can find examples of each type in the data, though the \emph{arguing} dimension of \emph{Exploring} is fairly infrequent due to explicit community rules: 

A major dimension of change is a shift from 

veteran users often engage in categorisation and explaination of the relationships between phenomena within the field of discourse

Veteran users also 

The seamlessness of transition between these kinds of meanings is readily apparent:

%example
%
The sharing of experiences and emotions doubles as a means of demonstrating the source of knowledge underlying \emph{advice}. That is, advice is often grounded in lay experience, rather than professional expertise.

This is different, however, from when veteran users do expounding. Categorisation of the relationship between symptoms, medications and healthcare professionals goes on absent any explicit mention of the source of the knowledge.

% WHY?


\section{Chapter summary}

In this chapter, I provided an account of OSG threads in terms of the systemic functional conceptualisation of genre. In the next chapter, I critically reflect on the case study.

