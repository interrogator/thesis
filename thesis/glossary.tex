%!TEX root = thesis.tex

\newglossaryentry{Forum}
{
  name=Forum,
  text=Forum,
  description={\emph{Forum} is the main term used to denote the site of the study, a large online \glslink{bipolar}{Bipolar Disorder} message board. \emph{Forum}, \emph{board}, \emph{community}, and \emph{group} are used interchangeably, just as they are by forum \glspl{member}.},
  sort=Forum,
  plural=Forums
}

\newglossaryentry{forum}
{
  name=forum,
  text=forum,
  description={Without an initial capital, \emph{forum}\slash \emph{forums} refers to online message boards---text-based \emph{modes} of \gls{CMC} where registered \glslink{member}{users} can create and reply to \glspl{thread}. The site of the investigation, the \emph{Bipolar \gls{Forum}}, is a fairly prototypical example.},
  sort=forum,
  plural=forums
}

\newglossaryentry{consumer}
{
  name=Consumer,
  text=consumer,
  description={\emph{Consumer} is used to refer to people who utilise healthcare services, ranging from consultations with healthcare professionals to reading information about health online. The term is used in preferece to \emph{patient}, which refers more specifically to those who are undergoing ongoing treatment in formal institutions (hospitals, psychiatric hospitals, clinics, etc.), and which construes the consumer as a (passive) recipient of medical services, rather than a collaborator in the process of treatment. \emph{Patient} is also less accurate when referring to the members of the \gls{forum}, as some members are not pursuing treatment through formal medical channels, but are nonetheless consuming healthcare information},
  sort=consumer,
  plural=consumers
}

\newglossaryentry{consumercentred}
{
  name=Consumer-centredness,
  text=consumer-centred,
  description={Synonymous with \emph{patient-centredness}, though \emph{consumer} is used here for consistency. \emph{Consumer-centredness} refers to a model of medical treatment that seeks to increase the agency of healthcare \glspl{consumer} in their own journeys through healthcare institutions. Under this model, \glspl{consumer}' attitudes, beliefs and emotional needs are prioritised, and \glspl{consumer} are encouraged to collaborate in decision-making processes. Another key aim is the fostering of an ongoing dialogue between professional and consumer. Consumer-centredness can increase consumer satisfaction, adherence to treatment plans, and lead to better health outcomes. Noting the importance of successful interpersonal exchange between healthcare professional and healthcare consumer, Matthiessen (2013) prefers \emph{relationship-centredness}.},
  sort=consumer-centredness,
  plural=consumer-centredness
}

\newglossaryentry{member}
{
  name=Member,
  text=member,
  description={Somebody who has signed up and\slash or contributed to an online community. Most often, in this thesis, this refers to those who contribute to the \glslink{Forum}{Bipolar Forum}. \emph{Members}, \emph{users}, \emph{contributors} and \emph{participants} are used interchangeably.},
  sort=member,
  plural=members
}

\newglossaryentry{bipolar}
{
  name=Bipolar Disorder,
  text=bipolar disorder,
  description={\emph{Bipolar Disorder} is a mental disorder characterised by oscillation between manic and depressive states. The thesis is not concerned with the phenomenology of the illness itself, making a more specific definition unnecessary here.},
  sort=bipolar,
  plural=bipolar
}

\newglossaryentry{corpus}
{
  name=Corpus,
  text=corpus,
  description={A collection of linguistic text, which is almost always large and digitised.},
  sort=corpus,
  plural=corpora
}

\newglossaryentry{post}
{
  name=Post,
  text=post,
  description={A single message within a \gls{thread}. Can be used as a verb to describe the process of authoring\slash transmitting a message. \emph{Contribution} is sometimes used.},
  sort=post,
  plural=posts
}

\newglossaryentry{thread}
{
  name=Thread,
  text=thread,
  description={A `discussion' within the Forum, initiated by a single user. The \gls{forum} presents a list of \glspl{thread}, organised by date of last post.},
  sort=thread,
  plural=threads
}

\newglossaryentry{lexicogrammar}
{
  name=Lexicogrammar,
  text=lexicogrammar,
  description={The stratum of words and wordings in language. A single system, with grammar at the broad end and lexis as the delicate end.},
  sort=lexicogrammar,
  plural=lexicogrammar
}

\newglossaryentry{discourse-semantic}
{
  name=Discourse-semantics,
  text=discourse-semantic,
  description={The stratum of function and meaning in language. It encompasses (interpersonal) pragmatics and ideation, as well as text organisation, which is not treated in detail in this thesis.},
  sort=discourse-semantics,
  plural=discourse-semantics
}

\newglossaryentry{theme}
{
  name=theme,
  description={Uncapitalised, \emph{theme} is used to mean a salient, recurring kind of meaning being made in a text (as in thematic analysis).},
  sort=theme,
  plural=themes
}

\newglossaryentry{Theme}
{
  name=Theme,
  description={With an initial capital, \emph{Theme} refers to the leftmost group in a clause, as per the systemic functional grammar.},
  sort=theme2,
  plural=Themes
}

\newglossaryentry{THEME}
{
  name=\sctheme{},
  description={In small caps, \sctheme{} denotes the system incorporating \emph{Theme}, as per the systemic functional grammar.},
  sort=theme3,
  plural=THEMES
}

\newglossaryentry{mode}
{
  name=mode,
  text=mode,
  description={Uncapitalised, \emph{mode} refers to a constellation of medium factors that together realise a culturally recognised variety of \gls{CMC}. A web forum is therefore a mode of CMC, as is an instant messenger, a wiki talk page, a Skype video call, etc.},
  sort=mode1,
  plural=modes
}

\newglossaryentry{Mode}
{
  name=Mode,
  text=Mode,
  description={With an initial capital, \emph{Mode} refers to the systemic-functional register dimension, which broadly corresponds to \emph{the role played by language in the text}.},
  sort=mode2,
  plural=Modes
}

\newdualentry[short=OSG,name=Online support group]{OSG}{OSG}{Online support group}{Online support groups are online spaces in which people can exchange information about health related topics, and to offer social support.}{OSGs}
\newdualentry[short=HTML]{HTML}{HTML}{HyperText Markup Language}{The main markup language for creating webpages. The \gls{Forum} was downloaded as HTML. Texts and metadata were extracted from the HTML using \texttt{lxml}.}{HTML}
\newdualentry[short=XML]{XML}{XML}{Extensible Markup Language}{An HTML-like markup language designed to store arbitrary information in a human-readable and machine-readable format.}{XML}

\newdualentry[short=SFG]{SFG}{SFG}{Systemic Functional Grammar}{A detailed description of the discourse-semantics, lexicogrammar and phonology of a language that draws upon systemic\hyp{}functional categories and the fits within a systemic\hyp{}functional conceptualisation of language and context. Such grammars are functional\hyp{}semantic, in that the functions and meanings made by language are used as the basis for grammatical distinctions. In the case of this thesis, \emph{SFG} refers to the grammar of English developed chiefly by Halliday, as presented in Halliday and Matthiessen 2004 and elsewhere. The phonological\slash graphological stratum is not considered in this work .}{SFG}

\newdualentry[short=CL,name=Corpus linguistics]{CL}{CL}{Corpus Linguistics}{In this thesis, \emph{CL} primarily denotes Corpus Linguistics---a branch of linguistics centred on quantitative analysis of digitised texts. In related and overlapping literature, \emph{CL} may denote \emph{computational linguistics}. Because there is no clear line between corpus and computational linguistics, however, and because the methods used in this thesis in many respects blur boundaries between the two, references to \emph{CL} can be understood as acknowledging both corpus and computational theory and methods. When disambiguation is necessary, the full names are used.}{CL}




%%%%%%%%%%%%%%%%%%%%%%%%%%%%%%%%%%%%%%%%
%%%%%%% CANDIDATES FOR GLOSSARY %%%%%%%%
%%%%%%%%%%%%%%%%%%%%%%%%%%%%%%%%%%%%%%%%

% FORMAL HEALTHCARE INSTITUTION
% NEWCOMER/VETERAN MEMBER


